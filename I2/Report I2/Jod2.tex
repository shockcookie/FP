\documentclass[30pt,a4paper]{article}
% Dokumenten Typ, titelseite, Schriftgröße, Seitenformat
\PassOptionsToPackage{dvipsnames}{xcolor}
% Füge neue Farben hinzu (standart 5 farben oder so)
\usepackage[utf8]{inputenc}
% Kodierung
\usepackage[T1]{fontenc}
% Umlaute
\usepackage[english]{babel}
% Eingebundene Sprachen
\usepackage{graphicx}
% Einbinden von Grafiken
\usepackage{wrapfig}
% Text um kleine Grafiken herumsetzen
\usepackage{amsmath}
\usepackage{amsfonts}
\usepackage{amssymb}
% Mathe Symbole und Commands
\usepackage{mathtools}
% Verbessert ams Packete von oben
\usepackage{nicefrac}
% Schönere Brüche
\usepackage{tikz}
\usepackage{circuitikz}
\usepackage{tikz-cd}
% Tikz Stuff
\usepackage{enumerate}
% Bessere Aufzählungen
\usepackage{cancel}
% z.B Durchstreichen von Sachen
\usepackage[hidelinks]{hyperref}
\usepackage{cleveref}
% Links und Referenzen innerhalb des Dokuments
\usepackage{tcolorbox}
% Wunderschöne Farbige Boxen mit Überschriften
\usepackage{caption}
% Erstellen von captions innerhalb einer Minipage
\usepackage[margin=1in]{geometry}
% Änderung der Gestaltung einer Seite (Überschreibt \documentclass)
\usepackage{placeins}
% Mit Hilfe von \FloatBarrier floats einschränken
\usepackage{booktabs}
% Bei Tabellen wird kann anstelle von \hline \toprule, \midrule und \bottomrule verwendet werden etc.
\usepackage{wasysym}
% Fügt eine Reihe von Symbolen wie Männlich Weiblich dazu
\usepackage{url}
% Füge Problemlos urls ein
\usepackage{pdfpages}




\hbadness=99999 
% Löst ein Problem mit \hbox

\newcommand{\angstrom}{\mbox{\normalfont\AA}}
\newenvironment{Dtabular}[2][1] {\def\arraystretch{#1}\tabular{#2}}
{\endtabular}


\def\centerarc[#1](#2)(#3:#4:#5)% Syntax: [draw options] (center) (initial angle:final angle:radius)
{ \draw[#1] ($(#2)+({#5*cos(#3)},{#5*sin(#3)})$) arc (#3:#4:#5); }

\title{
	\large Fortgeschrittenes Physik Lab	SS19 \\[4mm]
	\textbf{\LARGE Experiment: Spectroscopy at the Iodine Molecule
	} \\[4mm]
	(Durchgeführt am: 10.10.19 bei Sarcevic, Nikolina) \\}
% Titel des Experiments
\author{Erik Bode, Damian Lanzenstiel \\ (Group 103)}
% Autoren

\begin{document}
	
	\begin{titlepage}
		\maketitle
		\vspace{2cm}
		\begin{abstract}
			The experiment is concerned with the iodine transition $X^1\Omega_{0g}^+\leftrightarrow B^3\Pi_{0u}^+$. In the first part the absorption spectrum was taken with a CCD-Spectrometer. Here the Birge-Sponer-Plot was used to get the vibration constants $\omega_ex_x=1.04\pm 0.06\,\text{cm}^{-1}$ and $\omega_e=135\pm 5\,\text{cm}^{-1}$. With these values the dissociation energy was calculated with two different methods resulting in $D_{e1}=4300\pm 400\,\text{cm}^{-1}$ and $D_{e2}=4340\pm 60\,\text{cm}^{-1}$. The excitation energy was calculated to be $T_e=15790\pm 60\,\text{cm}^{-1}$. The second part a monochromator was calibrated and used to measure the emission spectrum. Here the transitions $\nu'=6 \rightarrow \nu''=(4-12,14)$ were identified. Most of the values  measured in both parts of the experiment are compatible with the literature values.
		\end{abstract}
	\end{titlepage}
	\newpage
	\tableofcontents
	\newpage
	
	\section{Theorie}
	\subsection{Radioactive Decays}
	Radioactive Decays are spontaneous processes in which a unstable atomic nucleus transforms into another lighter one while emitting other particles. Typical forms of radioactive decay are the $\alpha$, $\beta+$ and the $\beta-$decay.\\
	During the $\alpha-$decay a helium nucleus is emitted, reducing the atomic number by two. This form of decay is mainly found in heavy nucleus.\\ During the $\beta+$decay a proton transforms into a neutron and emits a positron as well as a electron-neutrino, reducing the atomic number by one.
	$$p\rightarrow n+e^++v_e$$
	On the other hand the $\beta-$decay is the reverse. It transforms a neutron into a proton and emits a electron and a electron-antineutrino. This decay increases the atomic number.
	$$n\rightarrow p+e^-+\bar{v}_e$$
	Another form of decay is the Electron Capture (EC) or $\epsilon-$decay. This one is similar to the $\beta+$decay since it also transforms a proton into a neutron. The difference being, that here the proton captures a electron to transform. The emitted particle is a electron-neutrino.
	$$p + e^- \rightarrow n + \bar{v}_e$$
	The captured electron is mostly from the K-shell while the resulting hole in the shell is filled by electrons from the L-shell. The remaining energy is either emitted through a X-ray photon or a Auger-electron. An Auger-electron is an electron that got the energy of an electron filling the vacancy left by electron in a lower state. The Auger-electron is therefore ejected. \\
	Decays are often accompanied by a $\gamma-$decays. When a decay occurs the daughter nucleus is mostly left in an exited state. It then decays into the ground state emitting $\gamma$-rays.\\
	Another Process similar to the $\gamma$-decay is the internal conversion (IC). Here the energy of a decay into a lower state is transmitted without radiation. That means no real photon is created to transport the energy. The energy is directly absorbed by another electron from the shell and ejected.	
	\subsection{Interaction between Matter and $\gamma-$Photons}	
 	When $\gamma-$photons and matter interact this happens mostly in 3 different ways depending on the atomic number of the atoms in the matter, as well as the Energy $E_\gamma$ of the photons.
 	\begin{enumerate}
 		\item Photoelectric effect:\\
 		The photoelectric effect happens when a photon is absorbed by an electron inside the matter. The energy carried by the photon is turned into kinetic energy and frees the electron. The vacancy is filled by electrons from higher shells and the energy is emitted by an Auger-electron or X-ray.\\
 		This effect appears mostly by $E_\gamma<200$\,keV and an atomic number around 50.
 		\item Compton scattering:\\
 		Unlike the photoelectric effect the photons are not absorbed by the electrons in the matter. They give up a part of their energy and scatter at the electron.\\
 		The Compton scattering occurs by energies in the range of $200$\,keV$<E_\gamma<5$\,MeV and a atomic number similar to the photoelectric effect.
 		\item Pair Production:\\
 		Pair production is an effect that appears by an energy $E_\gamma$ over the critical one of $1.022$\,MeV. When a $\gamma$-quantum gets into the electromagnetic field of a nucleus or electron it can be converted into an electron positron pair. 
 		$$\gamma \rightarrow e^- + e^+$$
 		To create this pair the energy of $1.022$\,MeV is needed this is also the reason the pair production can't happen if the photon has less energy. The remaining energy is given as kinetic energy to the electron and positron. The positron annihilates with an electron shortly after it's creation into two $\gamma$-rays with each half $0.511$\,MeV.
 	\end{enumerate}



\subsection{Radioaktiver Zerfall}
Der beim radioaktiven Zerfall verwandelt sich ein Instabiler Kern in einen leichteren unter Emission von Teilchen. Es existieren drei verschiedene Arten von radioaktiven Zerfall: 
\begin{itemize}
 \item[$\alpha$] {Bei dieser Zerfallsart stößt der Kern einen Heliumkern (ohne Elektronen) aus. Die Veränderung folgt diesem Schema:
$_Z^AX \rightarrow _{Z-4}^{A-2}Y + _4^2He^{2+}$ 
}
 \item[$\beta^-$]{Bei dieser Zerfallsart zerfällt ein } 
 \item[$\beta^+$]{}
\end{itemize}
 

	\section{Setup and Procedure of the Experiment}
\subsection{Measuring the Absorption Spectrum}
First the Absorption Spectrum of $I_2$ is measured. For this the setup in figure \ref{figABS} is used.\par
A Halogen lamp is used as a source of light and parallelised by a lens and refracted into an iodine tube which contains a iodine gas. After this it is focused into an CCD-Spectrometer where the spectrum is measured. To read the spectrum at the computer the program SpectraSuite is used. For the measurement a integration time of $100$\,ms was chosen. The measurement itself was done six times with only one scan. After this we discovered the ability to take the mean of multiple scans. We chose $30$ scans and did another measurement.
\begin{figure}[ht]
	\begin{tikzpicture}
		\coordinate (a) at (-3,3);
		\coordinate (b) at (-2.25,1);
		\coordinate (c) at (1,1);
		\coordinate (d) at (2,1);
		\draw ($(a) + (0.5,0.5)$) -- ($(a) + (0.5,1)$) -- ($(a) + (1,1)$) -- ($(a) + (1,0.5)$) -- ($(a) + (0.5,0.5)$);
		\draw ($(a) + (0.25,-0.5)$) -- ($(a) + (1.25,-0.5)$);
		\draw[dashed] ($(a) + (0.75,0.5)$) -- (b) -- ($(a) + (1.5,-2)$);
		\draw ($(b) + (-0.5,0.5)$) -- ($(b) + (0.5,-0.5)$);
		\draw ($(c) + (0,0.5)$) -- ($(c) + (-2.5,0.5)$) -- ($(c) + (-2.5,-0.5)$) -- ($(c) + (0,-0.5)$) -- ($(c) + (0,0.5)$);
		\draw[dashed] (c) -- ($(c) + (1,0)$) -- ($(c) + (1,1.5)$);
		\draw ($(d) + (-0.5,-0.5)$) -- ($(d) + (0.5,0.5)$);
		\draw ($(d) + (0,1.5)$) -- ($(d) + (0.5,2)$) -- ($(d) + (-0.5,2)$) -- ($(d) + (0,1.5)$);
		\draw ($(d) + (0.5,1)$) -- ($(d) + (-0.5,1)$);
		
		\node at ($(a) + (0,0.75)$)  {$1$};
		\node at ($(a) + (0,-0.5)$)  {$2$};
		\node at ($(b) + (-0.25,-0.25)$) {$3$};
		\node at ($(c) + (-1.25,1)$) {$4$};
		\node at ($(d) + (0.25,-0.25)$) {$3$};
		\node at ($(d) + (0.75,1)$) {$5$};
		\node at ($(d) + (0.75,2)$) {$6$};
	\end{tikzpicture}
	\centering
	\caption[Experimental Setup 1]{Experimental Setup for measuring the absorption spectrum. $1$ Halogen Lamp, $2$ paralleling Lens, $3$ Mirror, $4$ Iodine Tube, $5$ focus lens, $6$ CCD-Spectrometer.}
	\label{figABS}
\end{figure}
\subsection{Calibration of the Monochromator}
Before measuring the emission spectrum with the monochromator it has to be calibrated. For this a mercury vapour lamp is used instead of the halogen lamp. Also the second mirror is switched with the lens so that it know can be focus into the monochromator. In figure \ref{figCalib} the new setup can be seen. Since the x-Axis of the monochromator can't be trusted the start point as well as the end point were noted. As a step width $1\,\frac{\text{\AA}}{\text{s}}$ were chosen while the entrance slit was set to $50\,\mu$m. With this the first measurement was done. In the first one it was noticed, that that one of the peaks overshot. That's why we changed the settings at the discriminator level. With that the calibration measurement was done.
\begin{figure}[ht]
		\begin{tikzpicture}
	\coordinate (a) at (-3,3);
	\coordinate (b) at (-2.25,1);
	\coordinate (c) at (1,1);
	\coordinate (d) at (2,1);
	\draw ($(a) + (0.5,0.5)$) -- ($(a) + (0.5,1)$) -- ($(a) + (1,1)$) -- ($(a) + (1,0.5)$) -- ($(a) + (0.5,0.5)$);
	\draw ($(a) + (0.25,-0.5)$) -- ($(a) + (1.25,-0.5)$);
	\draw[dashed] ($(a) + (0.75,0.5)$) -- (b) -- ($(a) + (1.5,-2)$);
	\draw ($(b) + (-0.5,0.5)$) -- ($(b) + (0.5,-0.5)$);
	\draw ($(c) + (0,0.5)$) -- ($(c) + (-2.5,0.5)$) -- ($(c) + (-2.5,-0.5)$) -- ($(c) + (0,-0.5)$) -- ($(c) + (0,0.5)$);
	\draw[dashed] (c) -- ($(c) + (1.4,0)$);
	\draw ($(c) + (1.5,0)$) -- ($(c) + (2,0.5)$) -- ($(c) + (2,-0.5)$) -- ($(c) + (1.5,0)$);
	\draw ($(c) + (0.75,0.5)$) -- ($(c) + (0.75,-0.5)$);
	
	\node at ($(a) + (0,0.75)$)  {$1$};
	\node at ($(a) + (0,-0.5)$)  {$2$};
	\node at ($(b) + (-0.25,-0.25)$) {$3$};
	\node at ($(c) + (-1.25,1)$) {$4$};
	\node at ($(c) + (0.75,-0.75)$) {$5$};
	\node at ($(c) + (2.25,0)$) {$6$};
	\end{tikzpicture}
	\centering
	\caption[Experimental Setup 2]{Experimental Setup for calibrating the monochromator via a mercury vapour lamp. $1$ Mercury Vapour Lamp, $2$ paralleling Lens, $3$ Mirror, $4$ Iodine Tube, $5$ focus Lens, $6$ Monochromator.}
	\label{figCalib}
\end{figure}
\newpage
\subsection{Emission Spectrum}
For the actually measurement of the Emission Spectrum the setup in figure \ref{figEMS} was used. Here the iodine gas gets excite directly with a laser. First the laser needs to be directed with a mirror into the tube which isn't drawn into the sketch. After the tube the emitted light gets focused with two lenses into the monochromator. Here it is important to note that the red circle that is visible is only a reflection of the lenses and not the real focus point. The real point is not visible but is in the middle of the red circle. \par
After calibrating the point and heating the tube a bit the resonance laser peak was measured. As start point a wavelength of $6320\,\text{\AA}$ and a step width of $1\,\frac{\text{\AA}}{\text{s}}$ was chosen. The entrance slit was left at $50\,\mu$m.
After this measurement was taken the fluorescence spectrum from $6400\,\text{\AA}$ to $8000\,\text{\AA}$. Here the slit with was changed to $360\,\mu$m. Since the measurement needed a while the step with was set to $2\,\frac{\text{\AA}}{\text{s}}$ to find a good value for the discriminator. For the later measurements it was set back to $1\,\frac{\text{\AA}}{\text{s}}$.
\begin{figure}[ht]
	\begin{tikzpicture}
	\coordinate (a) at (-0.5,1.5);
	\coordinate (b) at (-2.25,1);
	\coordinate (c) at (1,1);
	\coordinate (d) at (2,1);
	\draw[dashed] ($(a) + (0.75,0.5)$) -- ($(a) + (0.75,0)$); 
	\draw ($(a) + (0.5,0.5)$) -- ($(a) + (0.5,1)$) -- ($(a) + (1,1)$) -- ($(a) + (1,0.5)$) -- ($(a) + (0.5,0.5)$);
	\draw ($(c) + (0,0.5)$) -- ($(c) + (-2.5,0.5)$) -- ($(c) + (-2.5,-0.5)$) -- ($(c) + (0,-0.5)$) -- ($(c) + (0,0.5)$);
	\draw[dashed] (c) -- ($(c) + (1.4,0)$);
	\draw ($(c) + (1.5,0)$) -- ($(c) + (2,0.5)$) -- ($(c) + (2,-0.5)$) -- ($(c) + (1.5,0)$);
	\draw ($(c) + (0.5,0.5)$) -- ($(c) + (0.5,-0.5)$);
	\draw ($(c) + (1,0.5)$) -- ($(c) + (1,-0.5)$);
	
	\node at ($(a) + (0,0.75)$)  {$1$};
	%\node at ($(a) + (0,-0.5)$)  {$2$};
	%\node at ($(b) + (-0.25,-0.25)$) {$3$};
	\node at ($(c) + (-1.25,-1)$) {$2$};
	\node at ($(c) + (0.75,-0.75)$) {$3$};
	\node at ($(c) + (2.25,0)$) {$4$};
	\end{tikzpicture}
	\centering
	\caption[Experimental Setup 3]{Experimental Setup for measurement of the emission spectrum with the monochromator via a laser. $1$ Laser, $2$ Iodine Tube, $3$ focus Lenses, $4$ Monochromator.}
	\label{figEMS}
\end{figure}
	\section{Auswertung}
\subsection{Pockelseffekt}
\subsubsection{Methode Sägezahnspannung}
Zur Auswertung des Pockelseffekt werden die Elektrooptischen Konstanten auf zwei Arten bestimmt. \par
Für die Auswertung von Methode eins wurde eine Sägezahnspannung an den Pockelszellen angelegt diese und das Signal an der Photodiode wurden mit einem Oszilloskop in $15$ Datensätze aufgezeichnet und geplottet. Die Anpassungen der Kurven an die Daten erfolgte mit dem Python Packet \verb|scipy.optimize| über \verb|curve_fit|. Hierbei wurde für den Anstieg der Sägezahnspannung die Form einer Geraden gewählt. Für die Spannung der Photodiode wurde als erstes die Form eines Sinus verwendet. Hier ließ sich jedoch die Kurve mit diesem nicht gut anpassen (siehe Abbildung \ref{ProbelmSinus}). Aus diesem Grund wurde ein Polynom neunten Grades (siehe Gleichung \ref{Poli}) an die Kurve gefittet. Dieser ist zusammen mit dem Linearen Fit in Abbildung \ref{PoliBild} zu finden.
\begin{equation}
	f(x)=ax^9+bx^8+c x^7+d x^6+ex^5+f x^4+gx^3+hx^2+ix+g
	\label{Poli}
\end{equation}
\begin{figure}[ht]
	\includegraphics[scale=0.5]{Bild/V1_1}
	\centering
	\caption[Plot zu Versuchsteil 1]{\small Datenpunkte der ersten Messmethode mit Sägezahnspannung. In rot der Fit an die Steigung der Sägezahnkurve und in grün die Polynom Anpassung an die Datenpunkte der Photodiode.}
	\label{PoliBild}
\end{figure}
Von diesem wurde dann das Maximum und Minimum bestimmt. Mit den x-Werten dieser konnten nun die Spannungen der Sägezahnkurve bestimmt werden, indem sie in den Linearen Fit eingesetzt wurde:
\begin{equation}
	U_\text{Sägezahn}= mx_{\text{max/min}} + c
\end{equation}
Der Fehler auf die Spannung ergibt sich über Gaußsche Fehlerfortpflanzung durch Gleichung \ref{SägeSp}
\begin{equation}
	\sigma_{U_\text{Sägezahn}}=\sqrt{\left(mx_{\text{max/min}}\sigma_c\right)^2+\left((c + x_{\text{max/min}})\sigma_m\right)^2}
	\label{SägeSp}
\end{equation}
Nun können die beiden Werte der Spannung voneinander abgezogen werden und man erhält die Spannungsdifferenz $U_{\frac{\lambda}{2}}$. Der Fehler errechnet sich durch:
\begin{equation}
	\sigma_{U_{\lambda/2}}=\sqrt{\left(U_\text{max}\sigma_{U_{\text{min}}}\right)^2+\left(U_{\text{min}}\sigma_{U_{\text{max}}}\right)^2}
	\label{Errdif}
\end{equation}
Da $15$ Messreihen erstellt wurde wird nun der Gewichtete Mittelwert dieser Werte genommen. Für dies wird Gleichung \ref{WMean} verwendet, der Fehler gibt sich wie in Gleichung \ref{Er_WMean}.
\begin{equation}
\bar{x}_g=\frac{\sum_ig_ix_i}{\sum_ig_i} \qquad \text{with} \qquad g_i=\frac{1}{\sigma_i^2}
\label{WMean}
\end{equation}
\begin{equation}
\sigma_{\bar{x}_g}=\frac{1}{\sqrt{\sum_i\nicefrac{1}{\sigma_i^2}}}
\label{Er_WMean}
\end{equation}
Als Wert ergibt sich für die Spannung dadurch $$U_{\frac{\lambda}{2}}=2.1278\pm0.0024\,\text{V}$$
Um nun die Elektrooptische Konstante $r_{41}$ zu bestimmen wird Gleichung \ref{r41} verwendet. Die Spannung muss davor jedoch noch um den Faktor $100$ vergrößert werden, da sie im Versuch um diesen Faktor gedämpft wurde. Die Werte für die Rechnung sind in der Versuchsanleitung \cite{anleitung} zu finden und wurden als fehlerfrei betrachtet. Damit ergab sich über die 1.Methode ein Wert von $$r_{41_1}=(26.489\pm0.030)\,\frac{\text{pm}}{V}$$
Die Annahme die Werte als Fehler frei zu betrachten stellte sich als Problematisch heraus, da dadurch der relative Fehler mit $0.11\%$ sehr klein wurde. Deshalb wurden für die Länge und Dicke des Kristalls weitere Fehler wie im Staatsexamen\cite{staatsex_farpock} von $0.1\,$mm abgeschätzt. Der Fehler der neue Fehler auf den Koeffizienten $r_{41}$ wird mit Gleichung \ref{neu} bestimmt. Der Wert mit diesen Fehlern beträgt:
$$r_{41_{F1}}=(26.5\pm1.1)\,\frac{\text{pm}}{\text{V}}$$
\begin{equation}
	\sigma_{r_{41}}=\sqrt{\left(\frac{\sigma_da}{lU_{\lambda/2}}\right)^2+\left(\frac{\sigma_lda}{l^2U_{\lambda/2}}\right)^2+\left(\frac{\sigma_{U_{\lambda/2}}a}{lU_{\lambda/2}^2}\right)^2}
	\label{neu}
\end{equation}
\subsubsection{Methode Gleichspannung}
Die zweite Methode die Konstante zu bestimmen ist über Gleichspannung. Hierbei ist die Spannung die Differenz zwischen den beiden gemessenen Gleichspannungssignalen. Da sich diese sehr schlecht haben einstellen lassen wurden für positive so wie negative Spannung das Signal sechs mal gemessen. Aus diesen wir nun als erstes der Arithmetische Mittelwert bestimmt so wie dessen Fehler. Hierbei werden die Gleichungen \ref{Mean} und \ref{ErrMean} verwendet um den Fehler zu bestimmen.
\begin{equation}
\sigma_x = \frac{1}{n-1}\sum_{i=1}^{n}(x_i-\bar{x})^2
\label{Mean}
\end{equation}
\begin{equation}
	\sigma_{\bar{x}}=\frac{\sigma_x}{\sqrt{n}}
	\label{ErrMean}
\end{equation}
Mit dem Mittelwert der beiden Spannungen wird nun der die Differenz der beiden Bestimmt. Der Fehler berechnet sich hier wie bei der Gleichung \ref{Errdif}.
Damit ergibt sich als Wert für die Spannung $$U_{\frac{\lambda}{2}}=(255.60\pm0.30)\,\text{V}$$ 
Dies kann wie bei der ersten Methode in die Gleichung \ref{r41}
eingesetzt werden wodurch sich der Wert für die Elektrooptische Konstante von
$$r_{41_2}=(22.051\pm 0.026)\,\frac{\text{pm}}{\text{V}}$$
ergibt. Da bei beiden Werten der Relative Fehler mit $0.11\%$ sehr klein war und sich eine eindeutige Unverträglichkeit zeige, wurde wie bei Methode eins auch hier die Werte nochmal mit einem geschätzten Fehler auf Dicke und Länge des Kristalls von $0.1$\,mm bestimmt. Fehler auch hier mit Gleichung \ref{neu}.
$$r_{41_{F2}}=(22.1\pm 0.9)\,\frac{\text{pm}}{\text{V}}$$
\subsection{Faraday Effekt}
Für die Auswertung des Faraday Effekts soll der Material abhängige Verdetkonstante bestimmt werden. Hierfür wird als erstes die Eingestellte Stromstärke gegen den gemessenen Drehwinkel aufgetragen. Die mit \verb|curve_fit| angepasste Gerade ist in Abbildung \ref{Dreh} zu finden. Die Geradengleichung lautet:
$$\alpha=(2.642\pm0.021)\,\frac{\circ}{\text{A}}\cdot I + (0.70\pm0.06)^\circ$$
\begin{figure}[ht]
	\includegraphics[scale=0.5]{Bild/V2Lin}
	\centering
	\caption[Datenpunkte und Fit der Faraday Messung]{\small In rot die gemessenen Datenpunkte und in blau der lineare Fit an diese. Es wurden keine Fehler mit eingezeichnet da diese zu klein waren um sie sinnvoll darzustellen.}
	\label{Dreh}
\end{figure}
Nun wird einmal für eine reale und eine ideale Spule die Verdetkonstante bestimmt.\par
Für eine reale Spule kann man Gleichung \ref{real} nehmen. Hierbei ist $\frac{\alpha}{I}$ gerade die Steigung der Geraden und der Faktor $2556$ ein Wert der im Staatsexamen\cite{staatsex_farpock} mit der Gleichung \ref{realFormel} für eine reale Spule berechnet wurde.
\begin{equation}
	V_{real}=\frac{\alpha}{\text{A}\cdot 2556}
	\label{real}
\end{equation}
Damit ergibt sich ein Wert für die Konstante von $$V_{real}=(0.001034 \pm 0.000008)\,\frac{\circ}{\text{A}}$$
Für eine ideale Spule wird die Gleichung \ref{idealH} verwendet.
\begin{equation}
	V_{ideal}=\frac{\alpha}{\text{A}}\frac{L}{Nl}\approx \frac{\alpha}{\text{A}\cdot 3086}
	\label{idealH}
\end{equation}
Hierbei ist $L$ die Länge der Spule, $l$ die Länge des Materials in der Spule und $N$ die Anzahl der Windungen. Werte sind in der Versuchsanleitung zu finden\cite{anleitung}. Hier ergibt sich für die Verdetkonstante ein Wert von:
$$V_{ideal}=(0.000856 \pm 0.000007)\,\frac{\circ}{\text{A}}$$
Um diese mit der Angabe des Herstellers von $V_{HS}=0.05\,\frac{\text{min}}{\text{Oe cm}}$ zu vergleichen müssen die Werte umgerechnet werden dabei gilt:
$$1\,\text{A}=\frac{100}{79.59}\,\text{Oe cm}  \qquad \qquad 1\,\text{Grad}=60\,\text{min}$$
Damit ergeben sich die Werte von:
$$V_{real}=(0.0494 \pm 0.0004)\,\frac{\text{min}}{\text{Oe cm}}$$
$$V_{ideal}=(0.04088 \pm 0.00032)\,\frac{\text{min}}{\text{Oe cm}}$$
	\section{Diskussion of the Results}
\subsection{Absorption Spectrum}

	
	\section*{Tabellen}
	\listoftables
	\section*{Bilder}
	\listoffigures
	\section{Bibliograpy}
	\bibliographystyle{plain}
	\bibliography{Quellen}
	\addcontentsline{toc}{section}{References}
	\section{Anhang}
	\begin{figure}[ht]
	H
\end{figure}
\end{document}