\section{Durchführung des Versuches}
\subsection{Vorbereitung}
Es wurde damit begonnen, die benötigten Geräte einzuschalten. Besonders wichtig war es, die Wasserkühlung der Magnetspule für die Faraday Effekt Messungen frühzeitig einzuschalten. 
\subsection{Pockels Effekt}

\subsubsection{Methoden zur Bestimmung von $U_{\lambda/2}$}
Es gibt zwei Arten, $U_{\lambda /2}$ zu bestimmen. Für die erste wird eine an der Pockels Zelle angelegte Sinusspannung mit einer Gleichspannung verschoben, bis das Intensitätssignal des Laserstrahls sich verändert: Die Frequenz verdoppelt sich und die Amplitude verringert sich deutlich. Sobald eine Veränderung des Intensitätssignals eintritt, wurde entweder der untere oder obere Grenzwert von $U_{\lambda/2}$ bestimmt. Nachdem der zweite Grenzwert bestimmt wurde, kann $U_{\lambda/2}$ bestimmt werden:
\begin{equation}\label{sinus-methode}
U_{\lambda/2} = U_{\textrm{Max}} - U_{\textrm{Min}}
\end{equation}
$U_{\lambda/2}$ wurde außerdem mithilfe von einer Sägezahnspannung bestimmt: An Stelle der obigen Sinusspannung wurde eine Sägezahnspannung an die Pockels Zelle angelegt. Dies führt zu einem annähernd Sinusförmigen Intensitätsverlauf. Zum Zeitpunkt des Maximums und Minimums der Intensität wird nun die Spannung der Sägezahnspannung gemessen. Die Differenz der beiden Punkte wird analog zu \ref{sinus-methode} verwendet um $U_{\lambda/2}$ zu bestimmen.

\subsubsection{Bestimmung von $U_{\lambda/2}$}
Als erstes wurde $U_{\lambda/2}$ mittels der Sägezahnspannungsmethode bestimmt. Hierzu wurden 15 Messungen durchgeführt, dass am Ende der Mittelwert gebildet werden kann.\\
Zur darauffolgend wurden die Gleichstrommessungen durchgeführt.Es wurde abwechselnd eine Positive oder Negative Gleichspannung angelegt. Insgesamt wurden sechs solcher Paare aufgezeichnet.

\subsection{Faraday Effekt}
Für die Messungen des Faraday Effektes wurde der Strom zur Felderzeugung in $0.5\,$A schritten hoch geregelt. Hierbei wurde immer der Drehwinkel des lichtes mittels des Halbschattenpolarimeters bestimmt. Als $5\,$A erreicht wurden, wurde die Richtung des Feldes umgekehrt und die Messung wiederholt.\par
Nach Abschluss der Messreihen wurden die $2 \epsilon$ Messungen durchgeführt.