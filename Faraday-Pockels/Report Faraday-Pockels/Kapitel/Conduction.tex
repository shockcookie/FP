\section{Durchführung des Versuches}
\subsection{Pockels Effekt}
\subsubsection{Bestimmung von $U_{\lambda/2}$}
Es gibt zwei Arten, $U_{\lambda /2}$ zu bestimmen. Für die erste wird eine an der Pockels Zelle angelegte Sinusspannung mit einer Gleichspannung verschoben, bis das Intensitätssignal des Laserstrahls sich verändert: Die Frequenz verdoppelt sich und die Amplitude verringert sich deutlich. Sobald eine Veränderung des Intensitätssignals eintritt, wurde entweder der untere oder obere Grenzwert von $U_{\lambda/2}$ bestimmt. Nachdem der zweite Grenzwert bestimmt wurde, kann $U_{\lambda/2}$ bestimmt werden:
\begin{equation}\label{sinus-methode}
U_{\lambda/2} = U_{\textrm{Max}} - U_{\textrm{Min}}
\end{equation}

$U_{\lambda/2}$ wurde außerdem mithilfe von einer Sägezahnspannung bestimmt: An Stelle der obigen Sinusspannung wurde eine Sägezahnspannung an die Pockels Zelle angelegt. Dies führt zu einem annähernd Sinusförmigen Intensitätsverlauf. Zum Zeitpunkt des Maximums und Minimums der Intensität wird nun die Spannung der Sägezahnspannung gemessen. Die Differenz der beiden Punkte wird analog zu \ref{sinus-methode} verwendet um $U_{\lambda/2}$ zu bestimmen.