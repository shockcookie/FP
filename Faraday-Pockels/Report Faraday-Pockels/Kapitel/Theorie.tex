\section{Theoretische Grundlagen}
\subsection{Doppelbrechung}
Doppelbrechung ist eine Eigenschaft von optisch anisotropen Stoffen. In diesen ist die Ausbreitungsgeschwindigkeit abhängig von Richtung und Polarisation der durchdringenden Welle. Das führt dazu, dass die Welle in zwei Teilstrahlen aufgespalten wird.


\subsection{Pockels Effekt}
Der Pockels Effekt tritt nur in Kristallen ohne Symmetriezentren auf: Doppelbrechung wird durch Anlegen einer externen Spannung erreicht.
Die Begründung dafür is, dass die Permittität $\epsilon$ nicht konstant ist, sondern vom angelegten elektrischen Feld abhängig ist. 
Die Permittität ist definiert über $\epsilon = \frac{\d D}{\d E}$ wobei $D$ definiert ist wie folgt: 
\begin{equation}
D = aE +bE^2+cE^3+... \qquad \textrm{mit } a,b,c = const
\end{equation}
Daraus folgt
\begin{equation}\label{epsilon}
\epsilon = a + 2bE + 3cE^2 + ...
\end{equation}.\par
Der Brechungsindex $n$ ist abhängig von $\epsilon$, sodass eine Änderung des Magnetfeldes eine Änderung im Indexellypsoid hervorruft. Dies ist der elektrooptische Pockels Effekt. Hier ist vor allem der lineare Term von \ref{epsilon} ausschlaggebend.
\subsection{Aufbau der Pockels Zelle}
Die im Versuch verwendete Pockels Zelle besteht aus 4 Ammoniumdihydrogenphosphat (ADP, $NH_4H_4PO_4$) Kristallen, welche im $45^\circ$-Y-Cut vorliegen.\\
Der Indexellipsoid  des Kristalls bis zu ersten Ordnung ist wie folgt:
\begin{equation}
	\frac{x_1^2}{n_1^2} + 2 r_{41} x_2 E_1 x_3 + \frac{x_2^2}{n_1^2}+ 2 r_{41} x_1 E_2 x_3 + \frac{x_3^2}{n_3^2} +  2 r_{63} x_1 x_2 E_3 = 1
\end{equation}

Hierbei ist optische Achse im feldfreien Fall die $x_3$ Achse. 	Wenn ein Elektrisches Feld entlang der $x_1$ Achse angelegt ist, gilt das Folgende:
\begin{equation}
		\frac{x_1^2}{n_1^2} + 2 R_{41} x_2 E_1 x_3 + \frac{x_2^2}{n_1^2}+  \frac{x_3^2}{n_3^2} = 1
\end{equation}

Y-Cut: Koordinatenwechsel durch Drehung con $45^\circ$ um $x_1$ Achse:
\begin{equation}
	x_2 = \frac{1}{\sqrt{2}} \left(x_2^\prime x_3^\prime \right) \qquad
	x_3 = \frac{1}{\sqrt{2}} \left(x_2^\prime x_3^\prime \right) 
\end{equation}
Nach der Herleitung in der Versuchsanleitung \cite{anleitung} folgt nun für dem Brechungsindex der jeweiligen Polarisationskomponenten bei Lichteinfall der $x_2^\prime$ ($x_3^\prime$) Richtung 
\begin{equation}
n_{x_2^\prime} = \frac{n_x}{\sqrt{1+ r_{41}E_1 n_x^2}} \approx n_x + \frac{1}{2}r_{41}E_1 n_x^3
\end{equation}
In einem Kristall der Länge $l$ lautet die Phasenverschiebung nun
\begin{equation}
\omega t = \frac{2\pi}{\lambda}\left(n_1 - n_{x_2^\prime}\right)
\end{equation}
Weil die optische Achse und der k-Vektor in einem Winkel von $45^\circ$ zueinander stehen, trennt sich der eingehende Lichtstrahl in einen ordentlichen und außerordentlichen Strahl, welche durch einen zweiten ADP Kristall, der um $180^\circ$ zum ersten verdreht ist. So werden die strahlen wieder vereinigt. Ihre Phasenverschiebung aufgrund der unterschiedlichen Ausbreitungsgeschwindigkeiten ist nun 
\begin{equation}
\omega t = \frac{2\pi}{\lambda}\cdot2\cdot\left(n_1 - n_{x_2^\prime}\right)
\end{equation}

Es ist zusätzlich noch die Natürliche Doppelbrechung vorhanden, welche um ein weiteres Kristallpaar mit Winkel $90^\circ$ verbaut ist. Dass der Pockels Effekt nicht gleichzeitig kompensiert wird, wurde das Elektrische Feld an diesen umgekehrt. Dieser Aufbau mit Strahlengang ist in Abbildung \ref{kristalle} erkennbar. Die resultierende Phasenverschiebung ist somit 
\begin{equation}
\omega t = \frac{4\pi}{\lambda}r_{41} E_1 n_x^3 l
\end{equation}
Bei einer Phasenverschiebung um $\pi$ und einem Elektrischen Feld von $E= \frac{U}{d}$ gilt somit:
\begin{equation}
\label{pockelsgleichung}
r_{41} = \frac{\lambda d}{4 l U_{\lambda/2}} \sqrt{\frac{1}{2}\left(\frac{1}{n_1^2}+\frac{1}{n_3^2}\right)}^3
\end{equation}

\begin{figure}[h]
	\centering
	\includegraphics[scale=0.7]{Bilder/kristalle}
	\caption[Aufbau der Pockels Zelle]{\small Das Bild zeigt den Aufbau der verwendeten Pockels Zelle. Die Elektroden für den Aufbau der elektrischen Felder sowie die Strahlenverläufe sind auch eingezeichnet.}
	\label{kristalle}
\end{figure}
