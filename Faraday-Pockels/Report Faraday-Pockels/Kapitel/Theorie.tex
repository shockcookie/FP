\section{Theoretische Grundlagen}
\subsection{Doppelbrechung}
Doppelbrechung ist eine Eigenschaft von optisch anisotropen Stoffen. In diesen ist die Ausbreitungsgeschwindigkeit abhängig von Richtung und Polarisation der durchdringenden Welle. Das führt dazu, dass die Welle in zwei Teilstrahlen aufgespalten wird.


\subsection{Pockels Effekt}
Der Pockels Effekt tritt nur in Kristallen ohne Symmetriezentren auf, bei welchen eine Doppelbrechung durch Anlegen einer externen Spannung auf. Die Begründung dafür is, dass die Permittität $\epsilon$ nicht konstant ist, sondern vom angelegten Magnetfeld abhängig ist. 
Die im Versuch verwendete Pockelszelle besteht aus 4 Ammoniumdihydrogenphosphat (ADP, $NH_4H_4PO_4$) Kristallen, welche im $45^\circ$-Y-Cut vorliegen.\\
Der Indexellipsoid  des Kristalls bis zu ersten Ordnung ist wie folgt:
\begin{equation}
	\frac{x_1^2}{n_1^2} + 2 r_{41} x_2 E_1 x_3 + \frac{x_2^2}{n_1^2}+ 2 r_{41} x_1 E_2 x_3 + \frac{x_3^2}{n_3^2} +  2 r_{63} x_1 x_2 E_3 = 1
\end{equation}

Hierbei ist optische Achse im feldfreien Fall die $x_3$ Achse. 	Wenn ein Elektrisches Feld entlang der $x_1$ Achse angelegt ist, gilt das Folgende:
\begin{equation}
		\frac{x_1^2}{n_1^2} + 2 R_{41} x_2 E_1 x_3 + \frac{x_2^2}{n_1^2}+  \frac{x_3^2}{n_3^2} = 1
\end{equation}

Y-Cut: Koordinatenwechsel durch Drehung um $x_1$ Achse:
\begin{equation}
	x_2 = \frac{1}{\sqrt{2}} \left(x_2^\prime x_3^\prime \right) \qquad
	x_3 = \frac{1}{\sqrt{2}} \left(x_2^\prime x_3^\prime \right) 
\end{equation}

