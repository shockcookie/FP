\documentclass[30pt,a4paper]{article}
% Dokumenten Typ, titelseite, Schriftgröße, Seitenformat
\PassOptionsToPackage{dvipsnames}{xcolor}
% Füge neue Farben hinzu (standart 5 farben oder so)
\usepackage[utf8]{inputenc}
% Kodierung
\usepackage[T1]{fontenc}
% Umlaute
\usepackage[german]{babel}
% Eingebundene Sprachen
\usepackage{graphicx}
% Einbinden von Grafiken
\usepackage{wrapfig}
% Text um kleine Grafiken herumsetzen
\usepackage{amsmath}
\usepackage{amsfonts}
\usepackage{amssymb}
% Mathe Symbole und Commands
\usepackage{mathtools}
% Verbessert ams Packete von oben
\usepackage{nicefrac}
% Schönere Brüche
\usepackage{tikz}
\usepackage{circuitikz}
\usepackage{tikz-cd}
% Tikz Stuff
\usepackage{enumerate}
% Bessere Aufzählungen
\usepackage{cancel}
% z.B Durchstreichen von Sachen
\usepackage[hidelinks]{hyperref}
\usepackage{cleveref}
% Links und Referenzen innerhalb des Dokuments
\usepackage{tcolorbox}
% Wunderschöne Farbige Boxen mit Überschriften
\usepackage{caption}
% Erstellen von captions innerhalb einer Minipage
\usepackage[margin=1in]{geometry}
% Änderung der Gestaltung einer Seite (Überschreibt \documentclass)
\usepackage{placeins}
% Mit Hilfe von \FloatBarrier floats einschränken
\usepackage{booktabs}
% Bei Tabellen wird kann anstelle von \hline \toprule, \midrule und \bottomrule verwendet werden etc.
\usepackage{wasysym}
% Fügt eine Reihe von Symbolen wie Männlich Weiblich dazu
\usepackage{url}
% Füge Problemlos urls ein
\usepackage{pdfpages}




\hbadness=99999 
% Löst ein Problem mit \hbox

\newenvironment{Dtabular}[2][1] {\def\arraystretch{#1}\tabular{#2}}
{\endtabular}

\title{
	\large Fortgeschrittenes Physik Lab	SS19 \\[4mm]
	\textbf{\LARGE Experiment: Faraday- und Pockelseffekt
	} \\[4mm]
	(Durchgeführt am: (01-02).09.19 bei Leena Diehl) \\}
% Titel des Experiments
\author{Erik Bode, Damian Lanzenstiel \\ (Group 103)}
% Autoren

\begin{document}
	
	\begin{titlepage}
		\maketitle
		\vspace{2cm}
		\begin{abstract}
		Im diesem Versuch geht es darum den Faraday und Pockelseffekt zu untersuchen. Dabei sollte einmal der Elektrooptische Koeffizient $r_{41}$ mit einer Anordnung von Pockelszellen bestimmt werden. Hierbei wurden zwei Methoden verwendet die zu den Ergebnissen von $r_{41_{F1}}=(26.5\pm1.1)\,\frac{\text{pm}}{V}$ und $r_{41_{F2}}=(22.1\pm 0.9)\,\frac{\text{pm}}{\text{V}}$ führte. Diese werden in der Diskussion mit dem Literaturwert von $r_{41}=23.4\,\frac{\text{pm}}{\text{V}}$ verglichen. Im Zweiten Teil des Experiments soll die Materialabhängige Verdetkonstante bestimmt werden. Hierbei wird einmal eine idealisierte Spule und einmal eine reale Spule zur Bestimmung der Konstante angenommen. 
		Im Vergleich mit den Herstellerangaben des fällt auf, dass nur der Wert der realen Spule von $V_{real}=(0.0494 \pm 0.0004)\,\frac{\text{min}}{\text{Oe cm}}$ mit den Hersteller Wert von $V_{ideal}=0.05\,\frac{\text{min}}{\text{Oe cm}}$  verträglich ist.
		\end{abstract}
	\end{titlepage}
	\newpage
	\tableofcontents
	\newpage
	\section{Theorie}
	\subsection{Radioactive Decays}
	Radioactive Decays are spontaneous processes in which a unstable atomic nucleus transforms into another lighter one while emitting other particles. Typical forms of radioactive decay are the $\alpha$, $\beta+$ and the $\beta-$decay.\\
	During the $\alpha-$decay a helium nucleus is emitted, reducing the atomic number by two. This form of decay is mainly found in heavy nucleus.\\ During the $\beta+$decay a proton transforms into a neutron and emits a positron as well as a electron-neutrino, reducing the atomic number by one.
	$$p\rightarrow n+e^++v_e$$
	On the other hand the $\beta-$decay is the reverse. It transforms a neutron into a proton and emits a electron and a electron-antineutrino. This decay increases the atomic number.
	$$n\rightarrow p+e^-+\bar{v}_e$$
	Another form of decay is the Electron Capture (EC) or $\epsilon-$decay. This one is similar to the $\beta+$decay since it also transforms a proton into a neutron. The difference being, that here the proton captures a electron to transform. The emitted particle is a electron-neutrino.
	$$p + e^- \rightarrow n + \bar{v}_e$$
	The captured electron is mostly from the K-shell while the resulting hole in the shell is filled by electrons from the L-shell. The remaining energy is either emitted through a X-ray photon or a Auger-electron. An Auger-electron is an electron that got the energy of an electron filling the vacancy left by electron in a lower state. The Auger-electron is therefore ejected. \\
	Decays are often accompanied by a $\gamma-$decays. When a decay occurs the daughter nucleus is mostly left in an exited state. It then decays into the ground state emitting $\gamma$-rays.\\
	Another Process similar to the $\gamma$-decay is the internal conversion (IC). Here the energy of a decay into a lower state is transmitted without radiation. That means no real photon is created to transport the energy. The energy is directly absorbed by another electron from the shell and ejected.	
	\subsection{Interaction between Matter and $\gamma-$Photons}	
 	When $\gamma-$photons and matter interact this happens mostly in 3 different ways depending on the atomic number of the atoms in the matter, as well as the Energy $E_\gamma$ of the photons.
 	\begin{enumerate}
 		\item Photoelectric effect:\\
 		The photoelectric effect happens when a photon is absorbed by an electron inside the matter. The energy carried by the photon is turned into kinetic energy and frees the electron. The vacancy is filled by electrons from higher shells and the energy is emitted by an Auger-electron or X-ray.\\
 		This effect appears mostly by $E_\gamma<200$\,keV and an atomic number around 50.
 		\item Compton scattering:\\
 		Unlike the photoelectric effect the photons are not absorbed by the electrons in the matter. They give up a part of their energy and scatter at the electron.\\
 		The Compton scattering occurs by energies in the range of $200$\,keV$<E_\gamma<5$\,MeV and a atomic number similar to the photoelectric effect.
 		\item Pair Production:\\
 		Pair production is an effect that appears by an energy $E_\gamma$ over the critical one of $1.022$\,MeV. When a $\gamma$-quantum gets into the electromagnetic field of a nucleus or electron it can be converted into an electron positron pair. 
 		$$\gamma \rightarrow e^- + e^+$$
 		To create this pair the energy of $1.022$\,MeV is needed this is also the reason the pair production can't happen if the photon has less energy. The remaining energy is given as kinetic energy to the electron and positron. The positron annihilates with an electron shortly after it's creation into two $\gamma$-rays with each half $0.511$\,MeV.
 	\end{enumerate}



\subsection{Radioaktiver Zerfall}
Der beim radioaktiven Zerfall verwandelt sich ein Instabiler Kern in einen leichteren unter Emission von Teilchen. Es existieren drei verschiedene Arten von radioaktiven Zerfall: 
\begin{itemize}
 \item[$\alpha$] {Bei dieser Zerfallsart stößt der Kern einen Heliumkern (ohne Elektronen) aus. Die Veränderung folgt diesem Schema:
$_Z^AX \rightarrow _{Z-4}^{A-2}Y + _4^2He^{2+}$ 
}
 \item[$\beta^-$]{Bei dieser Zerfallsart zerfällt ein } 
 \item[$\beta^+$]{}
\end{itemize}
 

	\section{Conduction of the experiment}
After the entrance exam the distances on the lid of the Dewar with the SQUID probe and the distance between the top of the Dewar and the position of the sample to later be able to compute the distance between the sample and the sensor. All distances were measured thee times to reduce the measurement inaccuracy. After that was finished, the Dewar was filled with the liquid nitrogen and the SQUID probe was placed inside to cool it down. While the sensor is cooling, the loop of the resistor measurements was measured from different angles because it is quite asymmetric. \par
Now, after approx. 15 minutes, the VCA and VCO settings in the control software of the SQUID were set as a calibration. They were modified so the SQUID signal has, as seen in figure \ref{cali_squid}, the characteristic differences from the usual sine function at the maxima and minima of the triangular reference voltage are as visible as possible.

\begin{figure}[ht]
\includegraphics[scale=0.5]{Bild/Eichung}
\centering
\caption[Picture of the calibration of the SQUID]{\small The figure shows the SQUID signal after being calibrated for the measurement. }
\label{cali_squid}
\end{figure}
\par
Now, after measuring the battery voltages, the measurements for the resistors were conducted. Starting with the smallest, four measurements of every resistor for each used motor speed were made. For the first resistor, the speed settings 10, 5 and 2 were used. During the measurements of the second resistor, it became clear that measurements of with the speed of 2 are not viable due to an increase in background interference. For the other resistors, only the speeds 10 and 5 were used and, also due to the increased background instabilities, only thee measurements each were made. After finishing the resistor measurements, the rotational speed of the motor settings 10 and 5 was measured over multiple rotations.\par
Now five different other samples were measured, each at a speed set to 10. The samples can be seen in figure \ref{samples} First a iron splinter, which worked well. Second a gold plate was tried, which did unfortunately not seem to have any measurable dipole moment. After taking two measurements, the signal suddenly disappeared and it seemed like, nothing was inside of the SQUID apparatus.Because there was a signal, the measurement should be retried later. The third sample was a magnet splinter, which also worked well.
After that, the gold sample was retried, and still no signal was measurable. Now, a stone was measured. After this, the gold sample was retried one last time, but it still showed no signal at all. As a last sample, a magnet was measured. It was chosen as the last sample, because it influences the detector so strong that for the rest of the day no other measurements can be made. \par
As the last measurement, the resistors were measured with the multimeter.

\begin{figure}[ht]
	\includegraphics[scale=0.1,angle=0]{Bild/samples}
	\centering
	\caption[Picture of the other samples]{\small The picture shows the other five samples used during the experiment. In the quadratic arranged samples, the top right one is the iron splinter, the top left is the magnet splinter. The bottom left one is the stone, the bottom right one is the magnet. To the right of the other samples, the gold sample is placed.}
	\label{samples}
\end{figure}
	\section{Auswertung}
\subsection{Pockelseffekt}
\subsubsection{Methode Sägezahnspannung}
Zur Auswertung des Pockelseffekt werden die Elektrooptischen Konstanten auf zwei Arten bestimmt. \par
Für die Auswertung von Methode eins wurde eine Sägezahnspannung an den Pockelszellen angelegt diese und das Signal an der Photodiode wurden mit einem Oszilloskop in $15$ Datensätze aufgezeichnet und geplottet. Die Anpassungen der Kurven an die Daten erfolgte mit dem Python Packet \verb|scipy.optimize| über \verb|curve_fit|. Hierbei wurde für den Anstieg der Sägezahnspannung die Form einer Geraden gewählt. Für die Spannung der Photodiode wurde als erstes die Form eines Sinus verwendet. Hier ließ sich jedoch die Kurve mit diesem nicht gut anpassen (siehe Abbildung \ref{ProbelmSinus}). Aus diesem Grund wurde ein Polynom neunten Grades (siehe Gleichung \ref{Poli}) an die Kurve gefittet. Dieser ist zusammen mit dem Linearen Fit in Abbildung \ref{PoliBild} zu finden.
\begin{equation}
	f(x)=ax^9+bx^8+c x^7+d x^6+ex^5+f x^4+gx^3+hx^2+ix+g
	\label{Poli}
\end{equation}
\begin{figure}[ht]
	\includegraphics[scale=0.5]{Bild/V1_1}
	\centering
	\caption[Plot zu Versuchsteil 1]{\small Datenpunkte der ersten Messmethode mit Sägezahnspannung. In rot der Fit an die Steigung der Sägezahnkurve und in grün die Polynom Anpassung an die Datenpunkte der Photodiode.}
	\label{PoliBild}
\end{figure}
Von diesem wurde dann das Maximum und Minimum bestimmt. Mit den x-Werten dieser konnten nun die Spannungen der Sägezahnkurve bestimmt werden, indem sie in den Linearen Fit eingesetzt wurde:
\begin{equation}
	U_\text{Sägezahn}= mx_{\text{max/min}} + c
\end{equation}
Der Fehler auf die Spannung ergibt sich über Gaußsche Fehlerfortpflanzung durch Gleichung \ref{SägeSp}
\begin{equation}
	\sigma_{U_\text{Sägezahn}}=\sqrt{\left(mx_{\text{max/min}}\sigma_c\right)^2+\left((c + x_{\text{max/min}})\sigma_m\right)^2}
	\label{SägeSp}
\end{equation}
Nun können die beiden Werte der Spannung voneinander abgezogen werden und man erhält die Spannungsdifferenz $U_{\frac{\lambda}{2}}$. Der Fehler errechnet sich durch:
\begin{equation}
	\sigma_{U_{\lambda/2}}=\sqrt{\left(U_\text{max}\sigma_{U_{\text{min}}}\right)^2+\left(U_{\text{min}}\sigma_{U_{\text{max}}}\right)^2}
	\label{Errdif}
\end{equation}
Da $15$ Messreihen erstellt wurde wird nun der Gewichtete Mittelwert dieser Werte genommen. Für dies wird Gleichung \ref{WMean} verwendet, der Fehler gibt sich wie in Gleichung \ref{Er_WMean}.
\begin{equation}
\bar{x}_g=\frac{\sum_ig_ix_i}{\sum_ig_i} \qquad \text{with} \qquad g_i=\frac{1}{\sigma_i^2}
\label{WMean}
\end{equation}
\begin{equation}
\sigma_{\bar{x}_g}=\frac{1}{\sqrt{\sum_i\nicefrac{1}{\sigma_i^2}}}
\label{Er_WMean}
\end{equation}
Als Wert ergibt sich für die Spannung dadurch $$U_{\frac{\lambda}{2}}=2.1278\pm0.0024\,\text{V}$$
Um nun die Elektrooptische Konstante $r_{41}$ zu bestimmen wird Gleichung \ref{r41} verwendet. Die Spannung muss davor jedoch noch um den Faktor $100$ vergrößert werden, da sie im Versuch um diesen Faktor gedämpft wurde. Die Werte für die Rechnung sind in der Versuchsanleitung \cite{anleitung} zu finden und wurden als fehlerfrei betrachtet. Damit ergab sich über die 1.Methode ein Wert von $$r_{41_1}=(26.489\pm0.030)\,\frac{\text{pm}}{V}$$
Die Annahme die Werte als Fehler frei zu betrachten stellte sich als Problematisch heraus, da dadurch der relative Fehler mit $0.11\%$ sehr klein wurde. Deshalb wurden für die Länge und Dicke des Kristalls weitere Fehler wie im Staatsexamen\cite{staatsex_farpock} von $0.1\,$mm abgeschätzt. Der Fehler der neue Fehler auf den Koeffizienten $r_{41}$ wird mit Gleichung \ref{neu} bestimmt. Der Wert mit diesen Fehlern beträgt:
$$r_{41_{F1}}=(26.5\pm1.1)\,\frac{\text{pm}}{\text{V}}$$
\begin{equation}
	\sigma_{r_{41}}=\sqrt{\left(\frac{\sigma_da}{lU_{\lambda/2}}\right)^2+\left(\frac{\sigma_lda}{l^2U_{\lambda/2}}\right)^2+\left(\frac{\sigma_{U_{\lambda/2}}a}{lU_{\lambda/2}^2}\right)^2}
	\label{neu}
\end{equation}
\subsubsection{Methode Gleichspannung}
Die zweite Methode die Konstante zu bestimmen ist über Gleichspannung. Hierbei ist die Spannung die Differenz zwischen den beiden gemessenen Gleichspannungssignalen. Da sich diese sehr schlecht haben einstellen lassen wurden für positive so wie negative Spannung das Signal sechs mal gemessen. Aus diesen wir nun als erstes der Arithmetische Mittelwert bestimmt so wie dessen Fehler. Hierbei werden die Gleichungen \ref{Mean} und \ref{ErrMean} verwendet um den Fehler zu bestimmen.
\begin{equation}
\sigma_x = \frac{1}{n-1}\sum_{i=1}^{n}(x_i-\bar{x})^2
\label{Mean}
\end{equation}
\begin{equation}
	\sigma_{\bar{x}}=\frac{\sigma_x}{\sqrt{n}}
	\label{ErrMean}
\end{equation}
Mit dem Mittelwert der beiden Spannungen wird nun der die Differenz der beiden Bestimmt. Der Fehler berechnet sich hier wie bei der Gleichung \ref{Errdif}.
Damit ergibt sich als Wert für die Spannung $$U_{\frac{\lambda}{2}}=(255.60\pm0.30)\,\text{V}$$ 
Dies kann wie bei der ersten Methode in die Gleichung \ref{r41}
eingesetzt werden wodurch sich der Wert für die Elektrooptische Konstante von
$$r_{41_2}=(22.051\pm 0.026)\,\frac{\text{pm}}{\text{V}}$$
ergibt. Da bei beiden Werten der Relative Fehler mit $0.11\%$ sehr klein war und sich eine eindeutige Unverträglichkeit zeige, wurde wie bei Methode eins auch hier die Werte nochmal mit einem geschätzten Fehler auf Dicke und Länge des Kristalls von $0.1$\,mm bestimmt. Fehler auch hier mit Gleichung \ref{neu}.
$$r_{41_{F2}}=(22.1\pm 0.9)\,\frac{\text{pm}}{\text{V}}$$
\subsection{Faraday Effekt}
Für die Auswertung des Faraday Effekts soll der Material abhängige Verdetkonstante bestimmt werden. Hierfür wird als erstes die Eingestellte Stromstärke gegen den gemessenen Drehwinkel aufgetragen. Die mit \verb|curve_fit| angepasste Gerade ist in Abbildung \ref{Dreh} zu finden. Die Geradengleichung lautet:
$$\alpha=(2.642\pm0.021)\,\frac{\circ}{\text{A}}\cdot I + (0.70\pm0.06)^\circ$$
\begin{figure}[ht]
	\includegraphics[scale=0.5]{Bild/V2Lin}
	\centering
	\caption[Datenpunkte und Fit der Faraday Messung]{\small In rot die gemessenen Datenpunkte und in blau der lineare Fit an diese. Es wurden keine Fehler mit eingezeichnet da diese zu klein waren um sie sinnvoll darzustellen.}
	\label{Dreh}
\end{figure}
Nun wird einmal für eine reale und eine ideale Spule die Verdetkonstante bestimmt.\par
Für eine reale Spule kann man Gleichung \ref{real} nehmen. Hierbei ist $\frac{\alpha}{I}$ gerade die Steigung der Geraden und der Faktor $2556$ ein Wert der im Staatsexamen\cite{staatsex_farpock} mit der Gleichung \ref{realFormel} für eine reale Spule berechnet wurde.
\begin{equation}
	V_{real}=\frac{\alpha}{\text{A}\cdot 2556}
	\label{real}
\end{equation}
Damit ergibt sich ein Wert für die Konstante von $$V_{real}=(0.001034 \pm 0.000008)\,\frac{\circ}{\text{A}}$$
Für eine ideale Spule wird die Gleichung \ref{idealH} verwendet.
\begin{equation}
	V_{ideal}=\frac{\alpha}{\text{A}}\frac{L}{Nl}\approx \frac{\alpha}{\text{A}\cdot 3086}
	\label{idealH}
\end{equation}
Hierbei ist $L$ die Länge der Spule, $l$ die Länge des Materials in der Spule und $N$ die Anzahl der Windungen. Werte sind in der Versuchsanleitung zu finden\cite{anleitung}. Hier ergibt sich für die Verdetkonstante ein Wert von:
$$V_{ideal}=(0.000856 \pm 0.000007)\,\frac{\circ}{\text{A}}$$
Um diese mit der Angabe des Herstellers von $V_{HS}=0.05\,\frac{\text{min}}{\text{Oe cm}}$ zu vergleichen müssen die Werte umgerechnet werden dabei gilt:
$$1\,\text{A}=\frac{100}{79.59}\,\text{Oe cm}  \qquad \qquad 1\,\text{Grad}=60\,\text{min}$$
Damit ergeben sich die Werte von:
$$V_{real}=(0.0494 \pm 0.0004)\,\frac{\text{min}}{\text{Oe cm}}$$
$$V_{ideal}=(0.04088 \pm 0.00032)\,\frac{\text{min}}{\text{Oe cm}}$$	
	\section{Zusammenfassung}
Wenn man die gemessenen Werte mit denen des Literatur Wertes von $119\,$ns mit der Formel \ref{Vergleich} vergleicht erhält man die in Tabelle \ref{VglTable} beschriebenen Werte. \\
\begin{equation}
t=\frac{\left\|a-b\right|}{\Delta a}
\label{Vergleich}
\end{equation}


%\begin{table}
%	\label{VglTable}
%	\begin{Dtabular}[1.1]{|c|c|c|}
%		\hline
%		Messreihe&Lebensdauer $\tau$[ns]&Vergleichswert\\
%		\hline
%		Abkühlen 1 bei $0^\circ$&$122.4\pm2.2)$&$1.5$\\
%		\hline
%		Abkühlen 1 bei $90^\circ$&$122.2\pm2.0$&$1.6$\\
%		\hline
%		Aufwärmen bei $0^\circ$&$102.8\pm1.3$&$1.4$\\
%		\hline
%		Abkühlen 2 bei $0^\circ$&$116.6\pm1.7$&$0.5$\\
%		\hline
%		Abkühlen 2 bei $90^\circ$&$118.1\pm1.8$&$12.5$\\
%		\hline
%	\end{Dtabular}
%\end{table}

\begin{center}
	\begin{table}[h]
		\centering
		\begin{tabular}{|c|c|c|}
			\hline
			Messreihe&Lebensdauer $\tau$[ns]&Vergleichswert\\
			\hline
			Abkühlen 1 bei $0^\circ$&$122.4\pm2.2)$&$1.5$\\
			\hline
			Abkühlen 1 bei $90^\circ$&$122.2\pm2.0$&$1.6$\\
			\hline
			Aufwärmen bei $0^\circ$&$102.8\pm1.3$&$1.4$\\
			\hline
			Abkühlen 2 bei $0^\circ$&$116.6\pm1.7$&$0.5$\\
			\hline
			Abkühlen 2 bei $90^\circ$&$118.1\pm1.8$&$12.5$\\
			\hline
		\end{tabular}
	\caption[Endergebnisse]{Vergleich der berechneten Lebensdauern mit dem Literaturwert}
	\label{VglTable}
	\end{table}
\end{center}




Man erkennt schnell, dass alle gemessenen Werte bis auf die Aufwärmmessung mit dem Literaturwert kompatibel sind. Die große Diskrepanz kann man dadurch erklären, dass für die Messung während des Aufwärmvorgangs nicht gewartet wurde bis die Temperatur des Thermometers mit der der Probe angeglichen hat. Dadurch ziehen wir einen großen systematischen Fehler mit welcher die Unverträglichkeit erklären könnte. \par 
Ein weiteres Problem ist sicher die geringe Anzahl an Messpunkten die wir in allen Messreihen hatten, wodurch sich natürlich unsere Ergebnisse verschlechtern. Bei einer erneuten Durchführung des Versuches wäre es daher Sinnvoll, deutlich weniger Zeit auf die Kalibrierung der Erdmagnetfeldkompensation zu verwenden und stattdessen längere Abkühlungsmessungen durchzuführen und bei der Erwärmungsmessung die Leistung des Kühlaggregates langsam zurückzufahren statt es auszuschalten.
	
	
	
	
	
	\section{Tabellen}
	\listoftables
	\section{Bilder}
	\listoffigures
	\section{Bibliograpy}
	\bibliographystyle{plain}
	\bibliography{Quellen}
	\addcontentsline{toc}{section}{Literatur}
	\section{Anhang}
	\begin{figure}[ht]
	H
\end{figure}
\end{document}