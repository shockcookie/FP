\section{Discussion}
In the experiment the magnetic field and dipole moment of loops with different resistors  had to be calculated with the help of data measured by the SQUID (see table \ref{Messwerte2}). This can be compared to the theoretical value of the loops which can be seen in table \ref{Messwerte1}. The strength of the magnetic field $B_z$ is compared with equation \ref{vgl}:
\begin{equation}
		t = \frac{|x_1-x_2|}{\sqrt{\sigma_x1^2+\sigma_x2^2}}
		\label{vgl}
\end{equation}\\
\begin{table}[ht]
	\begin{Dtabular}[1.1]{|c|c|c|c|c|c|}
		\hline
		&R1&R2&R3&R4&R5\\
		\hline
		t[$\sigma$]&$5.705$&$5.598$&$5.635$&$5.570$&$5.276$\\
		\hline
	\end{Dtabular}
	\centering
	\caption[Vergleich der Magnetfelder]{Comparison of the magnetic field values $B_z$ with equation \ref{vgl}}
	\label{VGL}
\end{table}
As can be seen in the table \ref{VGL} the values aren't compatible with each other. Looking at the values we see that they are different by a factor of 4. The reason for this big difference probably duo to a larger than expected error by the measurement of the distance between probe and SQUID sensor. Here the measurement wasn't very precise duo to the lack of a good way to measure it. The impact on the results is also increased to to its huge influence duo to the fact that the magnetic field is proportional to $\frac{1}{z^3}$ (see eq:\ref*{Bz}). That means that even small differences can make a huge difference in the results. Another reason for errors is the form of the loop. It wasn't very symmetrical to begin with which makes $A=\pi r^2$ in the equation \ref{Dipolmoment1} only a rough estimation.
The measurement with the SQUID is naturally also not perfect duo to the huge amount of outer magnetic field which influence the measurement. It was especially hard to measure the weaker magnetic fields since the background noise was even with the Lock-In filter quite high.\par
In the second part of the experiment different samples were measured. Here the problem of the noise can be seen very good at the sample of a stone see figure: \ref{Sto}. Here the signal is mostly hidden in noise so the fit is very also not very accurate. Also there is no value to compare these samples to we still see some expected values. For example is the B-field of the magnet very big compared to the other samples. The magnetic chips are also not to low compared to the others. The stone on the other side has the lowest field. If we compare the polar representations with the the ones of the loop its also noticeable, that the samples aren't as symmetrical any more. \par