\section{Zusammenfassung}
Wenn man die gemessenen Werte mit denen des Literatur Wertes von $119\,$ns mit der Formel \ref{Vergleich} vergleicht erhält man die in Tabelle \ref{VglTable} beschriebenen Werte.
\begin{equation}
	t=\frac{x_M-x_Literatur}{\sigma_x}
	\label{Vergleich}
\end{equation}
\begin{table}[ht]
	\begin{Dtabular}[1.1]{|c|c|c|}
		\hline
		Messreihe&Lebensdauer $\tau$[ns]&Vergleichswert $t$\\
		\hline
		Abkühlen 1 bei $0^\circ$&$(122.4\pm0.022)\,$ns&$1.5$\\
		\hline
		Abkühlen 1 bei $90^\circ$&$(122.2\pm2.0)\,$ns&$1.6$\\
		\hline
		Aufwärmen bei $0^\circ$&$(102.8\pm1.3)\,$ns&$12.5$\\
		\hline
		Abkühlen 2 bei $0^\circ$&$(116\pm1.7)\,$ns&$1.4$\\
		\hline
		Abkühlen 2 bei $90^\circ$&$(118.1\pm1.8)\,$ns&$0.5$\\
		\hline
	\end{Dtabular}
	\centering
	\caption[Ergebnisse]{Bestimmte Lebensdauern und Vergleich zum Literaturwert von $119$ns.}
	\label{VglTable}
\end{table}
\FloatBarrier
Man erkennt schnell, dass alle gemessenen Werte bis auf die Aufwärmmessung mit dem Literaturwert kompatibel sind. Die große Diskrepanz kann man dadurch erklären, dass für die Messung während des Aufwärmvorgangs nicht gewartet wurde bis die Temperatur des Thermometers mit der der Probe angeglichen hat. Dies hat zur folge, dass wir eine andere Temperatur an der Probe haben als wir gemessen haben. Dadurch ziehen wir einen großen systematischen Fehler mit welcher die Unverträglichkeit erklären könnte.\par Ein weiteres Problem ist sicher die geringe Anzahl an Messpunkten, die wir in den ersten zwei Messreihen hatten, wodurch sich natürlich unsere Messreihe verschlechtert haben.  Die letzte Messreihe wurde ausführlicher und Präziser aufgenommen etwas was man auch an dem guten Wert für $90^\circ$ von $t=0.5$ sehen kann. Trotzdem wären sicher mehr Messpunkte hilfreich gewesen um eine höhere Statistische Relevanz zu erhalten was aus Zeitgründen leider nicht möglich war.\par
Weitere Gründe für Fehler sind die nicht Präzise genug eingestellten Helmholtz-Spulen. Diese hätte man wieder mit etwas mehr Zeit noch genauer einstellen können wie man besonders bei der $45^\circ$ Messung erkennt welche sehr unsymmetrisch ausgefallen ist. Auch die Einstellungen des Polarisationsfilters sind möglicher Weise nicht perfekt gewählt oder haben sich während der Messung verschoben. Dies könnte z.B denn großen Ausreißer des letzten Messpunktes in Abbildung \ref{Abk2} erklären. 