\section{Zusammenfassung}
Wenn man die gemessenen Werte mit denen des Literatur Wertes von $119\,$ns mit der Formel \ref{Vergleich} vergleicht erhält man die in Tabelle \ref{VglTable} beschriebenen Werte. \\
\begin{table}
	\begin{Dtabular}[1.1]{|c|c|c|}
		\hline
		Messreihe&Lebensdauer $\tau$[ns]&Vergleichswert\\
		\hline
		Abkühlen 1 bei $0^\circ$&$(1.224\pm0.022)$&\\
		\hline
		Abkühlen 1 bei $90^\circ$&&\\
		\hline
		Aufwärmen bei $0^\circ$&&\\
		\hline
		Abkühlen 2 bei $0^\circ$&&\\
		\hline
		Abkühlen 2 bei $90^\circ$&&\\
		\hline
	\end{Dtabular}
\end{table}
Man erkennt schnell, dass alle gemessenen Werte bis auf die Aufwärmmessung mit dem Literaturwert kompatibel sind. Die große Diskrepanz kann man dadurch erklären, dass für die Messung während des Aufwärmvorgangs nicht gewartet wurde bis die Temperatur des Thermometers mit der der Probe angeglichen hat. Dadurch ziehen wir einen großen systematischen Fehler mit welcher die Unverträglichkeit erklären könnte. \par Ein weiteres Problem ist sicher die geringe Anzahl an Messpunkten die wir in allen Messreihen hatten, wodurch sich natürlich unsere Messreihe verschlechtert. 