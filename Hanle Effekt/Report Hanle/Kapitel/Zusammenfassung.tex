\section{Zusammenfassung}
Wenn man die gemessenen Werte mit denen des Literatur Wertes von $119\,$ns mit der Formel \ref{Vergleich} vergleicht erhält man die in Tabelle \ref{VglTable} beschriebenen Werte. \\
\begin{equation}
t=\frac{\left\|a-b\right|}{\Delta a}
\label{Vergleich}
\end{equation}


%\begin{table}
%	\label{VglTable}
%	\begin{Dtabular}[1.1]{|c|c|c|}
%		\hline
%		Messreihe&Lebensdauer $\tau$[ns]&Vergleichswert\\
%		\hline
%		Abkühlen 1 bei $0^\circ$&$122.4\pm2.2)$&$1.5$\\
%		\hline
%		Abkühlen 1 bei $90^\circ$&$122.2\pm2.0$&$1.6$\\
%		\hline
%		Aufwärmen bei $0^\circ$&$102.8\pm1.3$&$1.4$\\
%		\hline
%		Abkühlen 2 bei $0^\circ$&$116.6\pm1.7$&$0.5$\\
%		\hline
%		Abkühlen 2 bei $90^\circ$&$118.1\pm1.8$&$12.5$\\
%		\hline
%	\end{Dtabular}
%\end{table}

\begin{center}
	\begin{table}[h]
		\centering
		\begin{tabular}{|c|c|c|}
			\hline
			Messreihe&Lebensdauer $\tau$[ns]&Vergleichswert\\
			\hline
			Abkühlen 1 bei $0^\circ$&$122.4\pm2.2)$&$1.5$\\
			\hline
			Abkühlen 1 bei $90^\circ$&$122.2\pm2.0$&$1.6$\\
			\hline
			Aufwärmen bei $0^\circ$&$102.8\pm1.3$&$1.4$\\
			\hline
			Abkühlen 2 bei $0^\circ$&$116.6\pm1.7$&$0.5$\\
			\hline
			Abkühlen 2 bei $90^\circ$&$118.1\pm1.8$&$12.5$\\
			\hline
		\end{tabular}
	\caption[Endergebnisse]{Vergleich der berechneten Lebensdauern mit dem Literaturwert}
	\label{VglTable}
	\end{table}
\end{center}




Man erkennt schnell, dass alle gemessenen Werte bis auf die Aufwärmmessung mit dem Literaturwert kompatibel sind. Die große Diskrepanz kann man dadurch erklären, dass für die Messung während des Aufwärmvorgangs nicht gewartet wurde bis die Temperatur des Thermometers mit der der Probe angeglichen hat. Dadurch ziehen wir einen großen systematischen Fehler mit welcher die Unverträglichkeit erklären könnte. \par 
Ein weiteres Problem ist sicher die geringe Anzahl an Messpunkten die wir in allen Messreihen hatten, wodurch sich natürlich unsere Ergebnisse verschlechtern. Bei einer erneuten Durchführung des Versuches wäre es daher Sinnvoll, deutlich weniger Zeit auf die Kalibrierung der Erdmagnetfeldkompensation zu verwenden und stattdessen längere Abkühlungsmessungen durchzuführen und bei der Erwärmungsmessung die Leistung des Kühlaggregates langsam zurückzufahren statt es auszuschalten.