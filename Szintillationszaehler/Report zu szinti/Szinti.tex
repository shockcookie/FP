\documentclass[30pt,a4paper]{article}
% Dokumenten Typ, titelseite, Schriftgröße, Seitenformat
\PassOptionsToPackage{dvipsnames}{xcolor}
% Füge neue Farben hinzu (standart 5 farben oder so)
\usepackage[utf8]{inputenc}
% Kodierung
\usepackage[T1]{fontenc}
% Umlaute
\usepackage[german]{babel}
% Eingebundene Sprachen
\usepackage{graphicx}
% Einbinden von Grafiken
\usepackage{wrapfig}
% Text um kleine Grafiken herumsetzen
\usepackage{amsmath}
\usepackage{amsfonts}
\usepackage{amssymb}
% Mathe Symbole und Commands
\usepackage{mathtools}
% Verbessert ams Packete von oben
\usepackage{nicefrac}
% Schönere Brüche
\usepackage{tikz}
\usepackage{circuitikz}
\usepackage{tikz-cd}
% Tikz Stuff
\usepackage{enumerate}
% Bessere Aufzählungen
\usepackage{cancel}
% z.B Durchstreichen von Sachen
\usepackage[hidelinks]{hyperref}
\usepackage{cleveref}
% Links und Referenzen innerhalb des Dokuments
\usepackage{tcolorbox}
% Wunderschöne Farbige Boxen mit Überschriften
\usepackage{caption}
% Erstellen von captions innerhalb einer Minipage
\usepackage[margin=1in]{geometry}
% Änderung der Gestaltung einer Seite (Überschreibt \documentclass)
\usepackage{placeins}
% Mit Hilfe von \FloatBarrier floats einschränken
\usepackage{booktabs}
% Bei Tabellen wird kann anstelle von \hline \toprule, \midrule und \bottomrule verwendet werden etc.
\usepackage{wasysym}
% Fügt eine Reihe von Symbolen wie Männlich Weiblich dazu
\usepackage{url}
% Füge Problemlos urls ein
\usepackage{pdfpages}




\hbadness=99999 
% Löst ein Problem mit \hbox

\newenvironment{Dtabular}[2][1] {\def\arraystretch{#1}\tabular{#2}}
{\endtabular}

\title{
	\large Fortgeschrittenes Physik Lab	SS19 \\[4mm]
	\textbf{\LARGE Experiment: Szintillationszähler
	} \\[4mm]
	(Durchgeführt am: (07.-08).10.19 bei Patrick Scholer) \\}
% Titel des Experiments
\author{Erik Bode, Damian Lanzenstiel \\ (Group 103)}
% Autoren

\begin{document}
	
	\begin{titlepage}
		\maketitle
		\vspace{2cm}
		\begin{abstract}
			Während dieses Versuches wurden mehre NIM-Module in Funktionsweise und Signalform betrachtet. Anschließend sind Spektren von $^{22}Na$, $^{60}Co$, $^{152}Eu$ und $^{228}Th$ aufgenommen worden. Bei den ersten drei Isotopen gelang die Identifikation der erwarteten gemessenen Energien gut, bei Thorium stellte sich dies herausfordernder dar. Anschließend wurde die Winkelabhängigkeit der Vernichtungsphotonen ausgelöst durch Paarbildung gemessen.
		\end{abstract}
	\end{titlepage}
	\newpage
	\tableofcontents
	\listoftables
	\listoffigures
	\newpage
	\section{Theorie}
	\subsection{Radioactive Decays}
	Radioactive Decays are spontaneous processes in which a unstable atomic nucleus transforms into another lighter one while emitting other particles. Typical forms of radioactive decay are the $\alpha$, $\beta+$ and the $\beta-$decay.\\
	During the $\alpha-$decay a helium nucleus is emitted, reducing the atomic number by two. This form of decay is mainly found in heavy nucleus.\\ During the $\beta+$decay a proton transforms into a neutron and emits a positron as well as a electron-neutrino, reducing the atomic number by one.
	$$p\rightarrow n+e^++v_e$$
	On the other hand the $\beta-$decay is the reverse. It transforms a neutron into a proton and emits a electron and a electron-antineutrino. This decay increases the atomic number.
	$$n\rightarrow p+e^-+\bar{v}_e$$
	Another form of decay is the Electron Capture (EC) or $\epsilon-$decay. This one is similar to the $\beta+$decay since it also transforms a proton into a neutron. The difference being, that here the proton captures a electron to transform. The emitted particle is a electron-neutrino.
	$$p + e^- \rightarrow n + \bar{v}_e$$
	The captured electron is mostly from the K-shell while the resulting hole in the shell is filled by electrons from the L-shell. The remaining energy is either emitted through a X-ray photon or a Auger-electron. An Auger-electron is an electron that got the energy of an electron filling the vacancy left by electron in a lower state. The Auger-electron is therefore ejected. \\
	Decays are often accompanied by a $\gamma-$decays. When a decay occurs the daughter nucleus is mostly left in an exited state. It then decays into the ground state emitting $\gamma$-rays.\\
	Another Process similar to the $\gamma$-decay is the internal conversion (IC). Here the energy of a decay into a lower state is transmitted without radiation. That means no real photon is created to transport the energy. The energy is directly absorbed by another electron from the shell and ejected.	
	\subsection{Interaction between Matter and $\gamma-$Photons}	
 	When $\gamma-$photons and matter interact this happens mostly in 3 different ways depending on the atomic number of the atoms in the matter, as well as the Energy $E_\gamma$ of the photons.
 	\begin{enumerate}
 		\item Photoelectric effect:\\
 		The photoelectric effect happens when a photon is absorbed by an electron inside the matter. The energy carried by the photon is turned into kinetic energy and frees the electron. The vacancy is filled by electrons from higher shells and the energy is emitted by an Auger-electron or X-ray.\\
 		This effect appears mostly by $E_\gamma<200$\,keV and an atomic number around 50.
 		\item Compton scattering:\\
 		Unlike the photoelectric effect the photons are not absorbed by the electrons in the matter. They give up a part of their energy and scatter at the electron.\\
 		The Compton scattering occurs by energies in the range of $200$\,keV$<E_\gamma<5$\,MeV and a atomic number similar to the photoelectric effect.
 		\item Pair Production:\\
 		Pair production is an effect that appears by an energy $E_\gamma$ over the critical one of $1.022$\,MeV. When a $\gamma$-quantum gets into the electromagnetic field of a nucleus or electron it can be converted into an electron positron pair. 
 		$$\gamma \rightarrow e^- + e^+$$
 		To create this pair the energy of $1.022$\,MeV is needed this is also the reason the pair production can't happen if the photon has less energy. The remaining energy is given as kinetic energy to the electron and positron. The positron annihilates with an electron shortly after it's creation into two $\gamma$-rays with each half $0.511$\,MeV.
 	\end{enumerate}



\subsection{Radioaktiver Zerfall}
Der beim radioaktiven Zerfall verwandelt sich ein Instabiler Kern in einen leichteren unter Emission von Teilchen. Es existieren drei verschiedene Arten von radioaktiven Zerfall: 
\begin{itemize}
 \item[$\alpha$] {Bei dieser Zerfallsart stößt der Kern einen Heliumkern (ohne Elektronen) aus. Die Veränderung folgt diesem Schema:
$_Z^AX \rightarrow _{Z-4}^{A-2}Y + _4^2He^{2+}$ 
}
 \item[$\beta^-$]{Bei dieser Zerfallsart zerfällt ein } 
 \item[$\beta^+$]{}
\end{itemize}
 

	\section{Conduction of the experiment}
After the entrance exam the distances on the lid of the Dewar with the SQUID probe and the distance between the top of the Dewar and the position of the sample to later be able to compute the distance between the sample and the sensor. All distances were measured thee times to reduce the measurement inaccuracy. After that was finished, the Dewar was filled with the liquid nitrogen and the SQUID probe was placed inside to cool it down. While the sensor is cooling, the loop of the resistor measurements was measured from different angles because it is quite asymmetric. \par
Now, after approx. 15 minutes, the VCA and VCO settings in the control software of the SQUID were set as a calibration. They were modified so the SQUID signal has, as seen in figure \ref{cali_squid}, the characteristic differences from the usual sine function at the maxima and minima of the triangular reference voltage are as visible as possible.

\begin{figure}[ht]
\includegraphics[scale=0.5]{Bild/Eichung}
\centering
\caption[Picture of the calibration of the SQUID]{\small The figure shows the SQUID signal after being calibrated for the measurement. }
\label{cali_squid}
\end{figure}
\par
Now, after measuring the battery voltages, the measurements for the resistors were conducted. Starting with the smallest, four measurements of every resistor for each used motor speed were made. For the first resistor, the speed settings 10, 5 and 2 were used. During the measurements of the second resistor, it became clear that measurements of with the speed of 2 are not viable due to an increase in background interference. For the other resistors, only the speeds 10 and 5 were used and, also due to the increased background instabilities, only thee measurements each were made. After finishing the resistor measurements, the rotational speed of the motor settings 10 and 5 was measured over multiple rotations.\par
Now five different other samples were measured, each at a speed set to 10. The samples can be seen in figure \ref{samples} First a iron splinter, which worked well. Second a gold plate was tried, which did unfortunately not seem to have any measurable dipole moment. After taking two measurements, the signal suddenly disappeared and it seemed like, nothing was inside of the SQUID apparatus.Because there was a signal, the measurement should be retried later. The third sample was a magnet splinter, which also worked well.
After that, the gold sample was retried, and still no signal was measurable. Now, a stone was measured. After this, the gold sample was retried one last time, but it still showed no signal at all. As a last sample, a magnet was measured. It was chosen as the last sample, because it influences the detector so strong that for the rest of the day no other measurements can be made. \par
As the last measurement, the resistors were measured with the multimeter.

\begin{figure}[ht]
	\includegraphics[scale=0.1,angle=0]{Bild/samples}
	\centering
	\caption[Picture of the other samples]{\small The picture shows the other five samples used during the experiment. In the quadratic arranged samples, the top right one is the iron splinter, the top left is the magnet splinter. The bottom left one is the stone, the bottom right one is the magnet. To the right of the other samples, the gold sample is placed.}
	\label{samples}
\end{figure}
	\section{Analyse der Energiespektren}
Da leider nicht alle Messungen über die selbe Zeit aufgenommen wurden, mussten diese über die Messdauer normiert werden, es wurden also immer die Zählraten anstatt der Counts verwendet. Diese Normierung wurde immer nach 
Berechnung der  statistischen Fehler des Datensatzes über die Formel \ref{zählfehler} durchgeführt. Die Fehler wurden nach der Formel \ref{fehlerfortp} fortgepflanzt.
\begin{equation}
\Delta_N = \sqrt{N}
\label{zählfehler}
\end{equation}
\begin{equation}
	\Delta_\frac{N}{t} = \frac{\Delta_N}{t} 
	\label{fehlerfortp}
\end{equation}
\subsection{Untergrund}
Zu Beginn wurde die Untergrundmessung ausgewertet, da diese benötigt wird um alle weiteren Messungen zu korrigieren. Es wurde statt einer langen Messung zwei kürzere aufgenommen, was durch addieren der Counts pro Bin behoben wurde.
Für die resultierende Datenreihe wurden wie oben beschrieben die statistischen Fehler berechnet und die Normierung durchgeführt. Die zur Hintergrundkompensation genutzte Datenreihe ist in Abbildung \ref{untergrund} zu sehen.
\begin{figure}[h]
	\centering
	\includegraphics[scale=0.5]{Bilder/untergrund}
	\caption[Normalisierter Hintergrund]{\small Im Bild ist der Hintergrund Datensatz zu sehen. Es wurde die Zählrate pro Kanal aufgetragen. }
	\label{untergrund}
\end{figure}
\subsection{Energiekalibrierung}
Um den aktuellen Zusammenhang zwischen Kanälen des MCA und Energie der Detektierten $\gamma$ Photonen zu bestimmen, wurden die bekannten Peaks der Spektren von Natrium, Cobalt und Europium verwendet. Die so gewonnenen Zusammenhänge wurden auf den gesamten Bereich extrapoliert. Es wurden immer die statistischen Fehler in den Fits berücksichtigt. 
\subsubsection{Natrium}
Für das Spektrum der $^{22}Na$ Probe wurden zwei Peaks erwartet, die Intensität des mit niedrigerer Energie einen deutlich größer als die des höher energetischen Peaks. Genau dies wurde beobachtet, somit konnten erfolgreich die Kanal-Energie Zusammenhänge für den 511 und 1275 keV Peak bestimmt werden. Hierzu wurden zunächst Analog zur Hintergrundmessung die statistischen Fehler bestimmt und die Counts in die Zählrate umgerechnet. 
Anschließend wurde eine Gaußkurve (nach Gleichung \ref{gaussian}) mittels Python \cite{SciPy_Opti} gefittet. Die Mittelpunkte der Peaks ($\mu$), welche für die Kalibrierung verwendet wurden sind der Abbildung \ref{natrium} zu entnehmen. 
\begin{figure}[h]
	\centering
	\includegraphics[scale=0.5]{Bilder/Natrium}
	\caption[Natriumspektrum mit Peaks]{\small Im Bild ist das Natrium Spektrum zu sehen. Es wurde die Zählrate pro Kanal aufgetragen und die Fits eingezeichnet. Zusätzlich sind die $\mu$ Werte der gefitteten Kurven eingetragen.}
	\label{natrium}
\end{figure}
\begin{equation}
f(x,\mu, \sigma, A, B) = B + A \cdot e ^{-\frac{(x - \mu) ^ 2}{2 \cdot \sigma ^ 2}}
\label{gaussian}
\end{equation}
\subsubsection{Cobalt}
Für das zur Kalibrierung verwendete Spektrum von $^{60}Co$ wurde analog zur Analyse des Natriumspektrums vorgegangen. Hier wurden zwei Peaks mit den Energien 1173 und 1333 keV erwartet. Diese konnten ohne größere Probleme gefunden und gefittet werden. Die relevanten Werte sind aus Abbildung \ref{cobalt} zu entnehmen. 
\begin{figure}[h]
	\centering
	\includegraphics[scale=0.5]{Bilder/Cobalt}
	\caption[Cobalt Spektrum mit Peaks]{\small Im Bild ist das Cobalt Spektrum zu sehen. Es wurde die Zählrate pro Kanal aufgetragen und die Fits eingezeichnet. Zusätzlich sind die $\mu$ Werte der gefitteten Kurven eingetragen.}
	\label{cobalt}
\end{figure}
\subsubsection{Europium}
Durch die höhere Komplexität des Zerfalls von $^{152}Eu$ wurden im gemessenen Spektrum viele Peaks aufgezeichnet. Mit der Natriummessung als Referenz konnten die ungefähren Kanäle für die Energien von 122 und 344 keV bestimmt werden und analog zu den anderen beiden Messungen die Auswertung durchgeführt werden. Die relevanten Werte sind aus Abbildung \ref{europium} zu entnehmen. Durch

\begin{figure}[h]
	\centering
	\includegraphics[scale=0.5]{Bilder/Europium}
	\caption[Europium Spektrum mit Peaks]{\small Im Bild ist das Europium Spektrum zu sehen. Es wurde die Zählrate pro Kanal aufgetragen und die Fits eingezeichnet. Zusätzlich sind die $\mu$ Werte der gefitteten Kurven eingetragen.}
	\label{europium}
\end{figure}
\subsubsection{Energieeichung}
Die in den vorherigen Abschnitten gewonnenen Energie-Kanal paare wurden nun aufgetragen und eine Linie wurde gefittet. Diesen Fit und die Parameter können in Abbildung \ref{energiekali} gefunden werden. Beim fit wurden die Fehler der Kanalnummer berücksichtigt, da die Energien ohne Fehler angenommen werden.

\begin{figure}[h]
	\centering
	\includegraphics[scale=0.5]{Bilder/Energieeichung}
	\caption[Energieeichung]{\small Hier wurden die Kanal-Energie Paare aufgetragen und eine Kurve gefittet. Die Funktion und deren Parameter sowie die Güte sind zusätzlich eingetragen. }
	\label{energiekali}
\end{figure}

\subsection{Thorium Spektrum}
Das $^{228}Th$ Spektrum ist viel komplexer als die vorherigen. Es wurden 9 Peaks gefittet, wobei die Abbildungen hierzu im Anhang zu finden sind. Der 3. Peak ist ein Doppelpeak, sodass nicht die übliche Funktion (\ref{gaussian}) sondern eine Kombination aus zwei solcher Funktionen (\ref{doppelgaus}) gefittet wurde. Es wurde auch weiterhin Python \cite{SciPy_Opti} verwendet. Es wurden weiterhin die Statistischen Fehler berücksichtigt, da die Fehler auf die Energie viel kleiner sind.
\begin{equation}
	f(x,\mu,\mu_2,\sigma,\sigma_2,A,A_2,B) = 
	 A \cdot e ^{-\frac{(x - \mu) ^ 2}{2 \cdot \sigma ^ 2}} + 
	 A_2 \cdot e ^{-\frac{(x - \mu_2) ^ 2}{2 \cdot \sigma_2 ^ 2}} + 
	 B
	 \label{doppelgaus}
\end{equation}
In Abbildung \ref{thorium} ist das gesamte Thorium Spektrum in Logarithmischer Skala aufgetragen. Anschließend wurden kleinere Ausschnitte aufgetragen, dass die einzelnen Peaks besser erkennbar sind. Im Spektrum bis 500 keV (Abb. \ref{th_500}) wurden alle 5 Peaks gefittet. Im Spektrum von 500 bis 1000 keV (Abb. \ref{th_1000}) wurden alle 3 gut sichtbaren Peaks gefittet. Am Ende wurde noch der gut sichtbare Peak bei $\approx 3700$ keV gefittet. Die Energien der gefitteten Peaks sind auf den Abbildungen der jeweiligen Peaks in der Zuordnung zu finden.
Die Energie der Kanäle wurde mittels der Gleichung \ref{energie-kanal} bestimmt. Die Parameter $a$ und $b$ hier stammen aus der Energieeichung (Abb. \ref{energiekali}).
\begin{equation}
	E\left(Chn\right) = a \cdot Chn + b ; \qquad \Delta_E = \frac{\partial E}{\partial a} \Delta a + \frac{\partial E}{\partial Chn} \Delta Chn \frac{\partial E}{\partial b} \Delta b
	\label{energie-kanal}
\end{equation}

\begin{figure}[h]
	\centering
	\includegraphics[scale=0.7]{Bilder/Th_ganz}
	\caption[Gesamtes Thorium Spektrum]{\small Thorium Spektrum in log. Skala, sodass alle Peaks mit stark variierenden Zählraten sichtbar sind.}
	\label{thorium}
\end{figure}

\begin{figure}[h]
	\centering
	\includegraphics[scale=0.7]{Bilder/Th_1}
	\caption[Thorium Spektrum bis 500 keV]{\small Thorium Spektrum in lin. Skala bis 500 keV. Es sind die Peaks 1 bis 5 sichtbar, wobei Peak 3 ein Doppelpeak ist. Es wurde die Fehlerbalken zur Übersicht nicht bei allen Punkten eingezeichnet, die Energiefehler sind außerdem zu klein um dargestellt zu werden.}
	\label{th_500}
\end{figure}

\begin{figure}[h]
	\centering
	\includegraphics[scale=0.7]{Bilder/Th_2}
	\caption[Thorium Spektrum 500 bis 1000 keV]{\small Thorium Spektrum in lin. Skala von 500 bis 1000 keV. Es sind die Peaks 6 bis 8 sichtbar.
	Es wurde die Fehlerbalken zur Übersicht nicht bei allen Punkten eingezeichnet, die Energiefehler sind außerdem zu klein um dargestellt zu werden.}
	\label{th_1000}
\end{figure}
 \FloatBarrier
\subsection{Identifizierung der gemessenen Peaks}
Im Anschluss an die Bestimmung der Energien von den 10 verschiedenen Peaks (vgl. oben) wurden diese mit der Zerfallsreihe von Thorium verglichen um so die Übergänge zu Identifizieren bei welchen die gezählten $\gamma$ Photonen erzeugt wurden.
\subsubsection{Peak 1}
Peak 1 wurde bei einer Energie von $77.26\pm0.10\,$keV gefittet. 
Beim Blick auf Abbildung \ref{p1} ist auffällig, dass der Peak unsauber ist, als ob er bei ca. 72 keV einen weiteren Peak besitzt, was den gesamten Fit verzieht. Das echte Maximum von Peak 1 liegt außerdem bei einer deutlich höheren Energie als das Maximum des Fits. Deshalb wird es als sinnvoll empfunden, einen anderen Mittelpunkt und ein größerer Fehler auf diesen Wert angenommen: Der neue Wert liegt nun bei $80\pm5\,$keV.\par
Wahrscheinlich handelt es sich hier um den Übergang von Thorium zu Radium mit Energie 84.373 keV \cite{Thorium}. Von der Energie würde zusätzlich noch der Übergang Thorium zu Radium bei 74.4 keV, möglich sein, was vermutlich den Peak leicht zu einer niedrigeren Energie verschob.
\begin{figure}[h]
	\centering
	\includegraphics[scale=0.7]{Bilder/Anhang/P1}
	\caption[Thorium Peak 1]{\small Die verwendeten Daten von Peak 1, dessen Fit und der Mittelpunkt des Fits.}
	\label{p1}
\end{figure}
\FloatBarrier
\subsubsection{Peak 2}
Das Maximum von Peak 2 wurde bei einer Energie von $149.67\pm0.08\,$keV gefittet. Dies ist aus Abbildung \ref{p2}. Durch die hohe Streuung in der Umgebung des Maximums wird es für sinnvoll gehalten, den Fehler zu vergrößern: Somit gilt nun für den Mittelpunkt des Fits die Energie von $150\pm5\,$keV.\par
Bei dieser Energie liegt kein direkter Peak vor, aber es existieren die Übergänge 131.612 keV  und 166.41 keV  von Thorium zu Radium \cite{Thorium} . Deshalb wird vermutet, dass der gemessene Peak eine Kombination von beiden sei, welche wegen den Unterschiedlichen Wahrscheinlichkeiten der Übergänge nicht mittig zwischen ihnen liegt, sondern etwas näher an 166.41 keV.
\begin{figure}[h]
	\centering
	\includegraphics[scale=0.7]{Bilder/Anhang/P2}
	\caption[Thorium Peak 2]{\small Die verwendeten Daten von Peak 2, dessen Fit und der Mittelpunkt des Fits.}
	\label{p2}
\end{figure}
\FloatBarrier
\subsubsection{Peak 3}
Peak 3 war ein Kombinationspeak. Deshalb werden die beiden Komponenten von ihm einzeln als 3.1 und 3.2 betrachtet. Dies ist alles in Abbildung \ref{p3} zu sehen.\\
Das gefittete Maximum von Peak 3.1 liegt bei $237.62\pm0.10\,$keV. Da der Kombinationsfit in dieser Region deutlich über der Mehrheit der Punkte liegt, wurde es für sinnvoll erachtet den Fehler auf einen größeren Wert festzulegen: Für Peak 3.1 gilt nun $238\pm2\,$keV.\\
Das gefittete Maximum von Peak 3.2 liegt bei $269.88\pm0.06\,$keV. Da hier am Peak eine gewisse Streuung der Messpunkte vorliegt, wird auch ein vergrößerter Fehler angenommen: Somit gilt für Peak 3.2 nun eine Energie von
$270\pm1\,$keV. Vermutlich spielen hier außerdem Überlagerungen von Comptoneffekten der darüberlegenden Peaks eine Rolle, weshalb zusätzlich systematische Fehler anzunehmen sind. \cite{staatsex_szinti} \par
Für die Energie zu Peak 3.1 passt der Übergang von Blei zu Bismut \cite{Blei} mit einer Energie von 238.632 keV. Für Peak 3.2 passt der Übergang von Thallium zu Blei bei 277.37 keV \cite{Thallium} am Wahrscheinlichsten. 


\begin{figure}[h]
	\centering
	\includegraphics[scale=0.7]{Bilder/Anhang/P3}
	\caption[Thorium Peak 3]{\small Die verwendeten Daten von Peak 3, der gefittete Kombinationspeak, die einzelnen Peaks und ihre Mittelpunkte.}
	\label{p3}
\end{figure}
\FloatBarrier
\subsubsection{Peak 4}
Das gefittete Maximum von Peak 4 liegt bei $345\pm0.23\,$ keV, was Abbildung \ref{p4} zu entnehmen ist. Es ist relativ offensichtlich, dass hier das echte Maximum der Daten neben dem des Fits liegt, weshalb für die Energie von Peak 4 der Wert $350\pm3\,$ keV angenommen wird. Vermutlich spielen hier außerdem Überlagerungen von Comptoneffekten der darüberlegenden Peaks eine Rolle, weshalb zusätzlich systematische Fehler anzunehmen sind \cite{staatsex_szinti}.\par
Hier existert kein Zerfall mit ähnlicher Energie zu Peak 4. Der nächste Übergang wäre der von Bismut zu Thallium bei einer Energie von 327.94 keV \cite{Bismut}. 
\begin{figure}[h]
	\centering
	\includegraphics[scale=0.7]{Bilder/Anhang/P4}
	\caption[Thorium Peak 4]{\small Die verwendeten Daten von Peak 4, der gefittete Peak und sein Mittelpunkt.}
	\label{p4}
\end{figure}
\FloatBarrier
\subsubsection{Peak 5}
Das gefittete Maximum von Peak 5 liegt bei $409.99\pm0.07\,$keV nach Abbildung \ref{p5}. Durch die hohe Streuung der Messwerte sollte ein höherer Fehler auf diesen Wert Angenommen werden, sodass für den Peak der Wert $410\pm2\,$keV gilt.\par
Hier wurde wahrscheinlich der Übergang von Blei zu Bismut mit einer Energie von 415.27 keV \cite{Blei} gemessen. 

\begin{figure}[h]
	\centering
	\includegraphics[scale=0.7]{Bilder/Anhang/P5}
	\caption[Thorium Peak 5]{\small Die verwendeten Daten von Peak 5, der gefittete Peak und sein Mittelpunkt.}
	\label{p5}
\end{figure}
\FloatBarrier
\subsubsection{Peak 6}
Das gefittete Maximum von Peak 6 liegt bei $518.8\pm0.9\,$keV nach Abbildung \ref{p6}. Durch die hohe Streuung der Messwerte ist ein größerer Fehler auf diesen Wert angenommen worden, sodass für den Peak der Wert $519\pm5\,$keV gilt.\par
Wahrscheinlich wurde hier der 510.7 keV \cite{Thallium} Übergang von Thallium zu Blei gemessen.

\begin{figure}[h]
	\centering
	\includegraphics[scale=0.7]{Bilder/Anhang/P6}
	\caption[Thorium Peak 6]{\small Die verwendeten Daten von Peak 6, der gefittete Peak und sein Mittelpunkt.}
	\label{p6}
\end{figure}
\FloatBarrier
\subsubsection{Peak 7}
Das gefittete Maximum von Peak 7 liegt bei $591.04\pm0.31\,$keV nach Abbildung \ref{p7}. Durch die hohe Streuung der Messwerte ist ein größerer Fehler auf diesen Wert angenommen worden, sodass für den Peak der Wert von $591\pm5\,$keV gilt.\par
Wahrscheinlich wurde hier der 587.8 keV \cite{Thallium} Übergang von Thallium zu Blei gemessen.

\begin{figure}[h]
	\centering
	\includegraphics[scale=0.7]{Bilder/Anhang/P7}
	\caption[Thorium Peak 7]{\small Die verwendeten Daten von Peak 7, der gefittete Peak und sein Mittelpunkt.}
	\label{p7}
\end{figure}
\FloatBarrier
\subsubsection{Peak 8}
Das gefittete Maximum von Peak 8 liegt bei $839.44\pm0.15\,$keV nach Abbildung \ref{p8}. Durch die hohe Streuung der Messwerte wurde ein größerer Fehler auf diesen Wert angenommen, sodass für den Peak ein Wert von $839\pm5\,$keV gilt.\par
Wahrscheinlich wurde hier der 835.9 keV \cite{Thallium} von Thallium zu Blei gemessen.

\begin{figure}[h]
	\centering
	\includegraphics[scale=0.7]{Bilder/Anhang/P8}
	\caption[Thorium Peak 8]{\small Die verwendeten Daten von Peak 8, der gefittete Peak und sein Mittelpunkt.}
	\label{p8}
\end{figure}

\subsubsection{Peak 9}
Das gefittete Maximum von Peak 9 liegt bei $2614.1\pm0.5\,$keV nach Abbildung \ref{p9}. Durch die hohe Streuung der Messwerte wurde ein größerer Fehler auf diesen Wert angenommen, sodass für den Peak ein Wert von $2614\pm10\,$keV gilt.\par
Wahrscheinlich wurde hier der 2614.511 keV \cite{Thallium} Übergang von Thallium zu Blei gemessen.

\begin{figure}[h]
\centering
\includegraphics[scale=0.7]{Bilder/Anhang/P9}
\caption[Thorium Peak 9]{\small Die verwendeten Daten von Peak 9, der gefittete Peak und sein Mittelpunkt.}
\label{p9}
\end{figure}
\FloatBarrier
	\section{Auswertung der Winkelabhängigen Messung}
	\section{Diskussion der Ergebnisse}
\subsection{Energiespektren}
Während der Hintergrundmessung (Abb. \ref{untergrund}) viel nach dem ersten Peak, welcher durch die Messtechnik erklärbar ist, ein weiterer Peak mit der Spitze bei Kanal $4133.4\pm1.3$ erkennbar (Abb. \ref{untergrund_peak}).
\begin{figure}[h]
	\centering
	\includegraphics[scale=0.5]{Bilder/untergrund_peak}
	\caption[Peak im Hintergrund]{\small Im Bild ist der Peak im  Hintergrund Datensatz, dessen Fit und der Mittelpunkt des Fits zu sehen. Es wurde die Zählrate pro Kanal aufgetragen.}
	\label{untergrund_peak}
\end{figure}
Der Kanal im Peak wurde nach Gleichung \ref{energie-kanal} in den Kalibrierten Energiewert von $1463 \pm 6\,$keV, was in der Nähe des Übergangs von $^{40}K$ zu $^{40}Ca$ mit 1312.1 keV \cite{staatsex_szinti}. Nach der Formel \ref{vertrag} liegen diese jedoch bei einer Kompatibilität von 25 Standartabweichungen. Dies legt nahe, dass es sich um einen anderen Zefall handeln sollte. Dennoch ist dies die einzige Theorie, welche wir haben da alle im Versuch verwendeten Proben während der Messung stark abgeschirmt waren. Die Kalium Probe jedoch ist ein Pulver, welches im Versuch Lange Halbwertszeiten verwendet wird, welcher sich in räumlicher Nähe zum Versuchsaufbau befindet.\par
\begin{equation}
t = \frac{\left|x-y\right|}{\Delta_x}
\label{vertrag}
\end{equation}
Die Fits zu den Kalibriermessungen liefen Problemlos, die Resultierende Energieeichung des MCA lief auch gut, was am hohen $\chi^2$ Wert erkennbar ist.\par
Bei der Analyse des Thorium Spektrums fiel jedoch auf, dass obwohl der Fit zum jeweiligen Peak optisch nicht sehr gut passte, der Fehler auf den Parameter zum Maximum relativ klein war.
Bei der Zuordnung von den gefitteten Peaks zu den Übergängen der Thorium Zerfallsreihe viel auf, dass sehr viele Übergänge möglich waren, jedoch die meisten relativ unwahrscheinlich. Es ist dennoch nicht auszuschließen dass die gemessenen Übergänge durch zusätzlich die unwahrscheinlicheren Übergänge enthalten. Eine Analyse der 10 gefitteten Peaks des Thorium Spektrums mittels der Formel \ref{vertrag} ist in Tabelle \ref{Ergebnisse} zu erkennen.
\begin{table}
	\caption[Ergebnisse des Thorium Spektrums]{Die ermittelten Werte der Peaks des Thorium Spektrums sowie die Energien der zugeordneten Übergänge und deren Kompatibilität wurden in der nachfolgenden Tabelle aufgetragen.}
	\begin{tabular}{llll}
		\toprule
		{} & Gemessene Werte [keV] & Literaturwerte [keV] &                         Verträglichkeit \\
		\midrule
		Peak 1   &                80+/-5 &               84.373 &                                  0.8746 \\
		Peak 2   &               150+/-5 &    (131.612, 166.41) &  (3.7, 3.3) \\
		Peak 3.1 &           238.0+/-2.0 &              238.632 &                                   0.316 \\
		Peak 3.2 &           270.0+/-1.0 &               277.37 &                                    7.37 \\
		Peak 4   &           350.0+/-3.0 &               327.94 &                                 7.35333 \\
		Peak 5   &           410.0+/-2.0 &               415.27 &                                   2.635 \\
		Peak 6   &               519+/-5 &                510.7 &                                    1.66 \\
		Peak 7   &               591+/-5 &                587.8 &                                    0.64 \\
		Peak 8   &               839+/-5 &                835.9 &                                    0.62 \\
		Peak 9   &             2614+/-10 &              2614.51 &                                  0.0511 \\
		\bottomrule
	\end{tabular}
\label{Ergebnisse}
\end{table}
Es ist erkennbar, dass einige Werte relativ nahe an den Entsprechenden Literaturwerten liegen, andere jedoch weit von ihnen entfernt liegen. Dies ist insbesondere bei den Peaks 4 und 5 der Fall. Dies kann zum Teil durch Überlagerung von Comptoneffekten der darauffolgenden Peaks erklärt werden. Zusätzlich ist es sehr auffällig, dass die geschätzten Fehler groß sind, sodass die gemessenen Peaks teilweise sehr kleine Inkompatibilität haben. Dies ist vor allem bei Peak 9 der Fall, wo der gemessene Wert mit dem Literaturwert fast vollends übereinstimmt. 
\subsection{Winkelabhängige Messung}
Auch hier wurde die Kompatiblität des gemessenen Wertes von $-0.81\pm0.11\,$Grad mit dem erwarteten Wert von $0^\circ$ über die Formel \ref{vertrag} verglichen mit dem Ergebnis 7.4, wonach sie nicht kompatibel sind. Bei Betrachtung der Abbildung \ref{Winkelbild} fällt auf, dass im wichtigen Bereich die Datenpunkte vom Fit stark abweichen. Daraus kann man folgern, dass dieser Fit nicht repräsentativ für die Daten der Wichtigen Region ist. Bei Betrachtung der Daten einzeln fällt auf, dass mit Abstand die höchste Anzahl an Counts bei $0^\circ$ gemessen wurde, was darauf hinweist dass ein Problem mit dem Fit vorliegt.
	\section{Zusammenfassung}
Wenn man die gemessenen Werte mit denen des Literatur Wertes von $119\,$ns mit der Formel \ref{Vergleich} vergleicht erhält man die in Tabelle \ref{VglTable} beschriebenen Werte. \\
\begin{equation}
t=\frac{\left\|a-b\right|}{\Delta a}
\label{Vergleich}
\end{equation}


%\begin{table}
%	\label{VglTable}
%	\begin{Dtabular}[1.1]{|c|c|c|}
%		\hline
%		Messreihe&Lebensdauer $\tau$[ns]&Vergleichswert\\
%		\hline
%		Abkühlen 1 bei $0^\circ$&$122.4\pm2.2)$&$1.5$\\
%		\hline
%		Abkühlen 1 bei $90^\circ$&$122.2\pm2.0$&$1.6$\\
%		\hline
%		Aufwärmen bei $0^\circ$&$102.8\pm1.3$&$1.4$\\
%		\hline
%		Abkühlen 2 bei $0^\circ$&$116.6\pm1.7$&$0.5$\\
%		\hline
%		Abkühlen 2 bei $90^\circ$&$118.1\pm1.8$&$12.5$\\
%		\hline
%	\end{Dtabular}
%\end{table}

\begin{center}
	\begin{table}[h]
		\centering
		\begin{tabular}{|c|c|c|}
			\hline
			Messreihe&Lebensdauer $\tau$[ns]&Vergleichswert\\
			\hline
			Abkühlen 1 bei $0^\circ$&$122.4\pm2.2)$&$1.5$\\
			\hline
			Abkühlen 1 bei $90^\circ$&$122.2\pm2.0$&$1.6$\\
			\hline
			Aufwärmen bei $0^\circ$&$102.8\pm1.3$&$1.4$\\
			\hline
			Abkühlen 2 bei $0^\circ$&$116.6\pm1.7$&$0.5$\\
			\hline
			Abkühlen 2 bei $90^\circ$&$118.1\pm1.8$&$12.5$\\
			\hline
		\end{tabular}
	\caption[Endergebnisse]{Vergleich der berechneten Lebensdauern mit dem Literaturwert}
	\label{VglTable}
	\end{table}
\end{center}




Man erkennt schnell, dass alle gemessenen Werte bis auf die Aufwärmmessung mit dem Literaturwert kompatibel sind. Die große Diskrepanz kann man dadurch erklären, dass für die Messung während des Aufwärmvorgangs nicht gewartet wurde bis die Temperatur des Thermometers mit der der Probe angeglichen hat. Dadurch ziehen wir einen großen systematischen Fehler mit welcher die Unverträglichkeit erklären könnte. \par 
Ein weiteres Problem ist sicher die geringe Anzahl an Messpunkten die wir in allen Messreihen hatten, wodurch sich natürlich unsere Ergebnisse verschlechtern. Bei einer erneuten Durchführung des Versuches wäre es daher Sinnvoll, deutlich weniger Zeit auf die Kalibrierung der Erdmagnetfeldkompensation zu verwenden und stattdessen längere Abkühlungsmessungen durchzuführen und bei der Erwärmungsmessung die Leistung des Kühlaggregates langsam zurückzufahren statt es auszuschalten.

	\section{Bibliograpy}
	\bibliographystyle{plain}
	\bibliography{Quellen}
	\addcontentsline{toc}{section}{Literatur}
	\begin{figure}[ht]
	H
\end{figure}
\end{document}