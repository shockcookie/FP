\section{Theorie}
	\subsection{Radioactive Decays}
	Radioactive Decays are spontaneous processes in which a unstable atomic nucleus transforms into another lighter one while emitting other particles. Typical forms of radioactive decay are the $\alpha$, $\beta+$ and the $\beta-$decay.\\
	During the $\alpha-$decay a helium nucleus is emitted, reducing the atomic number by two. This form of decay is mainly found in heavy nucleus.\\ During the $\beta+$decay a proton transforms into a neutron and emits a positron as well as a electron-neutrino, reducing the atomic number by one.
	$$p\rightarrow n+e^++v_e$$
	On the other hand the $\beta-$decay is the reverse. It transforms a neutron into a proton and emits a electron and a electron-antineutrino. This decay increases the atomic number.
	$$n\rightarrow p+e^-+\bar{v}_e$$
	Another form of decay is the Electron Capture (EC) or $\epsilon-$decay. This one is similar to the $\beta+$decay since it also transforms a proton into a neutron. The difference being, that here the proton captures a electron to transform. The emitted particle is a electron-neutrino.
	$$p + e^- \rightarrow n + \bar{v}_e$$
	The captured electron is mostly from the K-shell while the resulting hole in the shell is filled by electrons from the L-shell. The remaining energy is either emitted through a X-ray photon or a Auger-electron. An Auger-electron is an electron that got the energy of an electron filling the vacancy left by electron in a lower state. The Auger-electron is therefore ejected. \\
	Decays are often accompanied by a $\gamma-$decays. When a decay occurs the daughter nucleus is mostly left in an exited state. It then decays into the ground state emitting $\gamma$-rays.\\
	Another Process similar to the $\gamma$-decay is the internal conversion (IC). Here the energy of a decay into a lower state is transmitted without radiation. That means no real photon is created to transport the energy. The energy is directly absorbed by another electron from the shell and ejected.	
	\subsection{Interaction between Matter and $\gamma-$Photons}	
 	When $\gamma-$photons and matter interact this happens mostly in 3 different ways depending on the atomic number of the atoms in the matter, as well as the Energy $E_\gamma$ of the photons.
 	\begin{enumerate}
 		\item Photoelectric effect:\\
 		The photoelectric effect happens when a photon is absorbed by an electron inside the matter. The energy carried by the photon is turned into kinetic energy and frees the electron. The vacancy is filled by electrons from higher shells and the energy is emitted by an Auger-electron or X-ray.\\
 		This effect appears mostly by $E_\gamma<200$\,keV and an atomic number around 50.
 		\item Compton scattering:\\
 		Unlike the photoelectric effect the photons are not absorbed by the electrons in the matter. They give up a part of their energy and scatter at the electron.\\
 		The Compton scattering occurs by energies in the range of $200$\,keV$<E_\gamma<5$\,MeV and a atomic number similar to the photoelectric effect.
 		\item Pair Production:\\
 		Pair production is an effect that appears by an energy $E_\gamma$ over the critical one of $1.022$\,MeV. When a $\gamma$-quantum gets into the electromagnetic field of a nucleus or electron it can be converted into an electron positron pair. 
 		$$\gamma \rightarrow e^- + e^+$$
 		To create this pair the energy of $1.022$\,MeV is needed this is also the reason the pair production can't happen if the photon has less energy. The remaining energy is given as kinetic energy to the electron and positron. The positron annihilates with an electron shortly after it's creation into two $\gamma$-rays with each half $0.511$\,MeV.
 	\end{enumerate}



\subsection{Radioaktiver Zerfall}
Der beim radioaktiven Zerfall verwandelt sich ein Instabiler Kern in einen leichteren unter Emission von Teilchen. Es existieren drei verschiedene Arten von radioaktiven Zerfall: 
\begin{itemize}
 \item[$\alpha$] {Bei dieser Zerfallsart stößt der Kern einen Heliumkern (ohne Elektronen) aus. Die Veränderung folgt diesem Schema:
$_Z^AX \rightarrow _{Z-4}^{A-2}Y + _4^2He^{2+}$ 
}
 \item[$\beta^-$]{Bei dieser Zerfallsart zerfällt ein } 
 \item[$\beta^+$]{}
\end{itemize}
 
