\documentclass[30pt,a4paper]{article}
% Dokumenten Typ, titelseite, Schriftgröße, Seitenformat
\PassOptionsToPackage{dvipsnames}{xcolor}
% Füge neue Farben hinzu (standart 5 farben oder so)
\usepackage[utf8]{inputenc}
% Kodierung
\usepackage[T1]{fontenc}
% Umlaute
\usepackage[english]{babel}
% Eingebundene Sprachen
\usepackage{graphicx}
% Einbinden von Grafiken
\usepackage{wrapfig}
% Text um kleine Grafiken herumsetzen
\usepackage{amsmath}
\usepackage{amsfonts}
\usepackage{amssymb}
% Mathe Symbole und Commands
\usepackage{mathtools}
% Verbessert ams Packete von oben
\usepackage{nicefrac}
% Schönere Brüche
\usepackage{tikz}
\usepackage{circuitikz}
\usepackage{tikz-cd}
% Tikz Stuff
\usepackage{enumerate}
% Bessere Aufzählungen
\usepackage{cancel}
% z.B Durchstreichen von Sachen
\usepackage[hidelinks]{hyperref}
\usepackage{cleveref}
% Links und Referenzen innerhalb des Dokuments
\usepackage{tcolorbox}
% Wunderschöne Farbige Boxen mit Überschriften
\usepackage{caption}
% Erstellen von captions innerhalb einer Minipage
\usepackage[margin=1in]{geometry}
% Änderung der Gestaltung einer Seite (Überschreibt \documentclass)
\usepackage{placeins}
% Mit Hilfe von \FloatBarrier floats einschränken
\usepackage{booktabs}
% Bei Tabellen wird kann anstelle von \hline \toprule, \midrule und \bottomrule verwendet werden etc.
\usepackage{wasysym}
% Fügt eine Reihe von Symbolen wie Männlich Weiblich dazu
\usepackage{url}
% Füge Problemlos urls ein



\hbadness=99999 
% Löst ein Problem mit \hbox



\title{
	\large Advanced Physics Lab	SS19 \\[4mm]
	\textbf{\LARGE Experiment: \emph{\color{red}Name of the Experiment}
	} \\[4mm]
	(conducted on: \emph{\color{red}Date} with \emph{\color{red}Name of the supervisor}) \\}
% Titel des Experiments
\author{Erik Bode, Damian Lanzenstiel \\ (Group 103)}
% Autoren

\begin{document}
	
	\begin{titlepage}
	\maketitle
	\vspace{2cm}
	\begin{abstract}
	\color{red}
	Abstract (max. 200 words)\\
	\color{black}
	Write about motivations, goals, and methods of the experiment
	Include the most important measurement results and uncertainties
	\end{abstract}
	\end{titlepage}
	\newpage
	
	\tableofcontents
	\listoffigures
	\listoftables
	\newpage
	
	\section{Introduction of theories and methodologies \color{red}(about 3-4 pages)}
	Introduction of topic / important theories, definitions, formulas, and concepts
	Concisely describe important concepts and formulas, including proper citations of original work
	\\
	\ \\
	Description of used methodologies
	Again using proper citations of original work and references to the relevant part of your protocol
	\section{Main Part / Results}
	Document well structured\\
	Concise description of scientific question addressed by each part of the experiment
	Description of setup and measurements (for each part)\\
	Include all relevant materials and lab equipment and illustrate important schematics
	Proper illustration and description of results (for each part)\\
	Use graphical representations of raw data including uncertainties and list all applied experimental
	parameter settings\\
	Detailed description of the data analysis (for each part)\\
	Discussion of uncertainties (for each part)\\
	In particular, carefully separate between statistical and systematical uncertainties and indicate those
	properly
	\section{Summary and concluding discussion \color{red}(about one page)}
	Summary of the results of the data analysis\\
	Give the most important results and include uncertainties\\
	Discussion of results (in reference to and in view of your introduction)\\
	Are measurements limited by statistics or systematics? How would you improve the methodologies in
	order to increase the measurement precision?\\
	\section{Bibliography / Supplements}
	Complete and valid bibliography\\
	You may cite work that can be accessed and checked by your reader. In general this is only true for
	published references. Thus, old protocols from you or other students are no valid reference. Internet
	sources are valid only if the author(s) is/are known in principle (Wikipedia usernames are not sufficient).\\
	Format of references\\
	E.g., author(s), title, journal/book/thesis, year\\
	Usage of citations\\
	Proper and sufficient citations in the protocol\\
	Attachment of lab notes\\
	Well documented, complete, and legible lab notes are original work of students. Data used for analysis
	agrees with data from lab notes. Lab notes were taken during the experiment, this has been confirmed by
	the assistant (signature!)
\end{document}