\section{Zusammenfassung und Diskussion zum Versuchsteil 3}
Im Versuchsteil 3 ging es um Halbleiterdetektoren. Hierbei wurde eine Silizium Diode und ein CdTe Kristall verwendet. Mit ihnen wurden die Energiespektren von $^ {57}$Co und $^{241}$Am aufgezeichnet. Mit diesen Spektren wurde eine Energieeichung durchgeführt und dann die Energieauflösung der einzelnen Peaks so, wie das Absorptionsverhältnis der beiden Detektoren bei den Unterschiedlichen Peaks bestimmt.\par
Wenn man sich die Eichungen anschaut scheinen die Messwerte sehr gut zueinander zu passen, es fällt jedoch auf, dass die geraden eine Abweichung der Steigung voneinander haben, was zeigt, dass eine individuelle Kalibrierung sinnvoll ist.\par

Wenn man sich die Werte für die Absorptionsverhältnisse anschaut (siehe Tabelle \ref{MesswerteV3_1}) fällt auf das es einen großen unterschied zwischen denen bei höheren Energien also den beiden $^{57}$Co Photopeaks der Wert geringer ist als bei der niedrigeren vom $59.5\,$keV Peak von Americium. Dies steht etwas im Widerspruch zu den Literaturwerten bei denen das Absorptionsverhältnis für höhere Energien größer ist. Gründe dafür können Abweichungen der Herstellerangaben sein oder ein großer Verlust an Signalstärke durch Ladungsrekombination verloren haben, dass sie dem ursprünglichen Peak nicht mehr zugeschrieben werden können. Auch wurde die Absorption in der Epoxid-Schicht so wie der Si02-Schicht der Silizium-Diode nicht mitberücksichtigt.\par
Bei Betrachtung der Energieauflösung in Tabelle \ref{MesswerteV3_2} fällt auf, dass die Auflösung des Silizium Detektors besser ist als die des CdTe Detektors. Auch scheint sich die Auflösung bei höheren Energien merklich zu verbessern. So scheinen sich die Photopeaks von Cobalt fast doppelt so gut auflösen wie beim $59.5$\,keV Peak von Americium.
