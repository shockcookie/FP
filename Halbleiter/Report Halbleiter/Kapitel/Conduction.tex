\section{Durchführung des Versuches}
Vor der mündlichen Abfrage wurde die Messung vom Cobalt Spektrum mit dem CdTe Kristall gestartet mit einer Messzeit von einer Stunde. Nach der der mündlichen Abfrage wurde der CdTe Kristall gegen eine Silizium Diode getauscht und das zweite Cobalt Spektrum wurde mit der Messzeit von einer Stunde aufgenommen. \par
Während die obige Messung lief wurde mit den Vorbereitungen Absorptions- und Transmissionsmessungen zu Silizium gestartet. Es wurde der Strahlengang eingestellt und vorläufige Parameter des Lock-in Verstärkers gefunden.\par
Nun wurde auch das Cobalt Spektrum mittels der Silizium Diode aufgenommen, weshalb die Probe auf Americium gewechselt wurde und dessen Spektrum aufgezeichnet wurde. Die Messzeit betrug wieder eine Stunde.\par
Nach der ersten Testmessung zu Transmission und Absorption wurden die finalen Verstärkereinstellungen vorgenommen. Es wurden zwei normale Messungen und drei Untergrundmessungen aufgezeichnet: Eine ohne die Silizium Probe, eine ohne das Gitter und eine mit abgedunkeltem Spalt. Hierzu wurde das Gitter von $-90^{\circ}$ bis $+90^{\circ}$ gedreht und immer die Position des Gitters und der Widerstand der Probe sowie die Spannung des Pyrodetektors aufgezeichnet.\par
Nach Abschluss der ersten Americium Messung wurde klar, dass die Probe in der falschen Orientierung auf den Detektor platziert wurde, die Zählrate war viel niedriger als erwartet. Nachdem die Probe richtig platziert wurde, konnte die Messung mit Dauer einer Stunde erneut gestartet werden.\par
Da die Absorptions- und Transmissionsmessungen nun für Silizium abgeschlossen sind, wurde der Aufbau für die Germanium Probe vorbereitet: Das Gitter, der Filter und die Probe wurden vertauscht. Nun wurde der Verstärker für die Germanium Probe angepasst. Leider gab es zunächst Probleme, sodass man die realen Maxima erst bei maximaler Verstärkung finden konnte. Alls die Fehlersuche sich hinzog wurde beschlossen sich aufzuteilen und die Haynes und Shockley Messungen parallel durchzuführen.\par
Zunächst wurde der Offset von Glasfaser zu Elektrode bestimmt, anschließend wurde der Versuchsaufbau in Betrieb genommen. Es wurde nun die minimale Spannung  gefunden, bei welcher vermutet wurde, dass eine Auswertung der Messung in Form einer angepassten Gaußkurve möglich ist. Dann wurde in $2\,$V, später $4\,$V Schritten die Spannung erhöht bis zur maximal möglichen Spannung von $48\,$V.\par
Nun war die Messung des Americium Spektrums mit der Silizium Diode abgeschlossen und sie wurde gegen den CdTe Kristall getauscht. Anschließend wurde die letzte Messung gestartet. \par
Für das Haynes und Shockley Experiment wurden nun die Abstandsmessungen durchgeführt. Hierbei wurde zunächst die maximale Spannung an der Probe angelegt, und der Abstand so lange erhöht bis zur maximalen Distanz, bei welcher es für möglich gehalten wurde, eine Auswertung durchzuführen. Diese lag bei $9\,$mm auf der angebrachten Skala. Der Abstand wurde immer um einen mm zwischen den gespeicherten Messungen reduziert, bis zum minimalen Abstand von $2\,$mm. Es wurde nun noch einmal der Offset zwischen Glasfaser und Elektrode bestimmt, welcher jetzt $3.6\,$mm betrug. \par
Währenddessen wurde der Strahlenverlauf der Absorptions- und Transmissionsmessungen angepasst, sodass hier auch die oben erwähnten Messungen aufgenommen werden konnten.