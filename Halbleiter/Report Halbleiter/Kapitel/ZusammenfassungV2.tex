\section{Zusammenfassung und Diskussion zum Versuchsteil 2}
Im Versuchsteil 2 ging es darum die Elektronenwolken innerhalb eines Halbleiters zu untersuchen. Hierfür wurden ein $3$cm langer Germanium Block genutzt. Zur Messung wurde einmal der Abstand zwischen Laser und Nadel verändert und einmal die am Germanium angelegte Spannung.
Die berechneten Werte sind die Lebensdauer, Beweglichkeit so die Diffusion der Wolken und sind in Tabelle \ref{MesswerteV2} zu finden. Zum Vergleich der Werte mit dem Literaturwert wurde die Gleichung \ref{vgl} verwendet. Die Werte sind in Tabelle \ref{VGLV2} notiert.
\begin{equation}
	t=\frac{x_{Mess}-x_{Literatur}}{\sigma_{x_{Messung}}}
	\label{vgl}
\end{equation}
\FloatBarrier
\begin{table}[ht]
	\begin{Dtabular}[1.1]{|c|c|c|}
		\hline
		&t bei konst. Abstand $[\sigma]$&t bei konst. Spannung $[\sigma]$\\
		\hline
		$\mu$ &$0.0$&$23.875$\\
		\hline
		$\tau$ &$546.5$&$39.1$\\
		\hline
		$D$ &$202.0$&$144.3$\\
		\hline
	\end{Dtabular}
	\centering
	\caption[Vergleichswerte V2]{Vergleichswerte der Berechneten Werte mit dem Literaturwerten über Gleichung \ref{vgl}}
	\label{VGLV2}
\end{table}
Wenn man sich diese Werte anschaut passen bis auf den Wert bei $\mu$ mit konstantem Abstand keine der Werte überein. Die große Abweichung bei der Lebenszeit kann man damit erklären, dass die freien Elektronen tiefer in der Germanium Probe erzeugt werden. Gittereffekte naher der Oberfläche können daher zu ein Grund für die starke Verkürzung der Lebenszeit sein. Wenn man sich die Diffusionskonstanten anschaut fällt auf, dass diese nicht einmal in der selben Größenordnung sind. Hier könnte der Grund ein Fehler in der Auswertung sein, welcher jedoch nicht gefunden worden ist. Ein weiterer Grund für die großen Unterschiede könnte auch sein, dass man den Gaußfit mit einer Gerade hätte überlagern sollen. Da dies nicht getan wurde kann dies zu systematischen Fehlern geführt haben die, die Unterschiede hervorrufen. Jedoch sollte die hierbei entstandene Diskrepanz nicht den Unterschied von mehreren Größenordnungen verursacht haben.
