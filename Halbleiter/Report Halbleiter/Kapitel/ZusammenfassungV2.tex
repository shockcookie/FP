\section{Zusammenfassung und Diskussion zum Versuchsteil 2}
Im Versuchsteil 2 ging es darum die Elektronenwolken innerhalb eines Halbleiters zu untersuchen. Hierfür wurden ein $3$cm langer Germanium Block genutzt. Zur Messung wurde einmal der Abstand zwischen Laser und Nadel verändert und einmal die am Germanium angelegte Spannung.
Die berechneten Werte sind die Lebensdauer, Beweglichkeit so die Diffusion der Wolken und sind in Tabelle \ref{MesswerteV2} zu finden. Zum Vergleich der Werte mit dem Literaturwert wurde die Gleichung \ref{vgl} verwendet. Die Werte sind in Tabelle \ref{VGLV2} notiert.
\begin{equation}
	t=\frac{x_{Mess}-x_{Literatur}}{\sigma_{x_{Messung}}}
	\label{vgl}
\end{equation}
\FloatBarrier
\begin{table}[ht]
	\begin{Dtabular}[1.1]{|c|c|c|}
		\hline
		&t $[\sigma]$ bei konst. Abstand&t $[\sigma]$ bei konst. Spannung\\
		\hline
		$\mu$ &$0.0$&$23.875$\\
		\hline
		$\tau$ &$546.5$&$39.1$\\
		\hline
		$D$ &$202.0$&$144.3$\\
		\hline
	\end{Dtabular}
	\centering
	\caption{text}
	\label{VGLV2}
\end{table}
