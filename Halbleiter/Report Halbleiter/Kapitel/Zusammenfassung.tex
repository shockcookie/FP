\section{Zusammenfassung und Diskussion zum Versuchsteil 1}
In Versuchsteil 1 ging es darum, die Bandlückenenergie von Silizium und Germanium zu ermitteln. Dies wurde durch die Aufzeichnung der Transmission und des Wiederstandes bei der Bestrahlung der Probe mit Photonen verschiedener Energien erreicht. Durch die oben genannte Methode konnten nun aus den aufgenommenen Datensätzen  die Bandlückenenergien bestimmt werden. In der obigen Analyse wurden die Mittelwerte der Absorptions-und Transmissions Energien noch nicht erwähnt, welche die Endergebnisse für die jeweiligen Rechnungen wären. Diese wurden nun mit in Tabelle \ref{ende_v1} eingetragen. 
Die Literaturwerte zu den Ergebnissen stammen aus der Versuchsanleitung \cite{anleitung}. Die Literaturwerte wurden über die Formel \ref{vgl} mit den berechneten Ergebnissen verglichen. 
\FloatBarrier
\begin{table}[ht]
	\centering
	\caption[Ergebnisse V1]{Ergebnisse der Bandlücke Energiebestimmung in Silizium (Si) und Germanium (Ge).}
	\label{ende_v1}
	\begin{tabular}{llll}
		\toprule
		{} & Mittelwerte der Messer & Literaturwert& Kompatiblität \\
		{} & gebnisse in ev & in eV & \\
		\midrule
		1. Si Messung links  &                          1.09+/-0.04 &                1.12 &      0.671224 \\
		1. Si Messung rechts &                          1.07+/-0.04 &                1.12 &        1.1568 \\
		2. Si Messung links  &                          1.14+/-0.05 &                1.12 &      0.418861 \\
		2. Si Messung rechts &                        1.035+/-0.031 &                1.12 &       2.71722 \\
		Ge Messung links     &                          0.66+/-0.04 &                0.66 &    0.00882199 \\
		Ge Messung rechts    &                        0.633+/-0.029 &                0.66 &       0.93153 \\
		\bottomrule
	\end{tabular}
\end{table}
\FloatBarrier
Man sieht, dass alle Ergebnisse außer dem rechten der zweiten Silizium Messung ihren Literaturwerten kompatibel sind. Bei Betrachtung der zugehörigen linearen Regressionen (Abbildung \ref{si_2_r}) wird deutlich, dass dies an falsch gewählten Bereichen liegen könnte. \par 
Alles in allem verlief die Durchführung des Versuches annehmbar, abgesehen davon dass zunächst vergessen wurde der Strahlengang für Germanium zu optimieren was die Durchführung beträchtlich verlängerte. Die Hintergrundmessungen, welche durch entfernen der Gitter statt die Blende komplett zu verschließen stellten sich als brauchbar heraus. Unangenehm fiel während der Auswertung auf, dass die Geschwindigkeit des Motors variierte, was sich in einer unterschiedlichen Winkeländerungsrate niederschlug.