\documentclass[30pt,a4paper]{article}
% Dokumenten Typ, titelseite, Schriftgröße, Seitenformat
\PassOptionsToPackage{dvipsnames}{xcolor}
% Füge neue Farben hinzu (standart 5 farben oder so)
\usepackage[utf8]{inputenc}
% Kodierung
\usepackage[T1]{fontenc}
% Umlaute
\usepackage[german]{babel}
% Eingebundene Sprachen
\usepackage{graphicx}
% Einbinden von Grafiken
\usepackage{wrapfig}
% Text um kleine Grafiken herumsetzen
\usepackage{amsmath}
\usepackage{amsfonts}
\usepackage{amssymb}
% Mathe Symbole und Commands
\usepackage{mathtools}
% Verbessert ams Packete von oben
\usepackage{nicefrac}
% Schönere Brüche
\usepackage{tikz}
\usepackage{circuitikz}
\usepackage{tikz-cd}
% Tikz Stuff
\usepackage{enumerate}
% Bessere Aufzählungen
\usepackage{cancel}
% z.B Durchstreichen von Sachen
\usepackage[hidelinks]{hyperref}
\usepackage{cleveref}
% Links und Referenzen innerhalb des Dokuments
\usepackage{tcolorbox}
% Wunderschöne Farbige Boxen mit Überschriften
\usepackage{caption}
% Erstellen von captions innerhalb einer Minipage
\usepackage[margin=1in]{geometry}
% Änderung der Gestaltung einer Seite (Überschreibt \documentclass)
\usepackage{placeins}
% Mit Hilfe von \FloatBarrier floats einschränken
\usepackage{booktabs}
% Bei Tabellen wird kann anstelle von \hline \toprule, \midrule und \bottomrule verwendet werden etc.
\usepackage{wasysym}
% Fügt eine Reihe von Symbolen wie Männlich Weiblich dazu
\usepackage{url}
% Füge Problemlos urls ein
\usepackage{pdfpages}



\hbadness=99999 
% Löst ein Problem mit \hbox

\newenvironment{Dtabular}[2][1] {\def\arraystretch{#1}\tabular{#2}}
{\endtabular}

\title{
	\large Fortgeschrittenes Physik Lab	SS19 \\[4mm]
	\textbf{\LARGE Experiment: Halbleiter
	} \\[4mm]
	(Durchgeführt am: 26-27.09.19 bei Marc Hauser) \\}
% Titel des Experiments
\author{Erik Bode, Damian Lanzenstiel \\ (Group 103)}
% Autoren

\begin{document}
	
	\begin{titlepage}
		\maketitle
		\vspace{2cm}
		\begin{abstract}
		Im Versuch Halbleiter ging es darum sich mit Halbleitern und ihren unterschiedlichen Anwendungen vertraut zu machen. Hierbei wird im ersten Versuchsteil die Absorption und Transmission Werte eines Silizium und eines Germanium Halbleiters untersucht. Im zweiten Teil geht es um die Elektronenwolken welche sich innerhalb eines Halbleiters bei Bestrahlung mit einem Laser bilden. Im letzten Teil wurden zwei Strahlende Proben ($^{57}$Co und $^{241}$Am) mit zwei verschiedenen Halbleiter Detektoren untersucht indem ihr Energiespektrum aufgezeichnet wurde. 
		\end{abstract}
	\end{titlepage}
	\newpage
	\tableofcontents
	\newpage
	\section{Theorie}
	\subsection{Radioactive Decays}
	Radioactive Decays are spontaneous processes in which a unstable atomic nucleus transforms into another lighter one while emitting other particles. Typical forms of radioactive decay are the $\alpha$, $\beta+$ and the $\beta-$decay.\\
	During the $\alpha-$decay a helium nucleus is emitted, reducing the atomic number by two. This form of decay is mainly found in heavy nucleus.\\ During the $\beta+$decay a proton transforms into a neutron and emits a positron as well as a electron-neutrino, reducing the atomic number by one.
	$$p\rightarrow n+e^++v_e$$
	On the other hand the $\beta-$decay is the reverse. It transforms a neutron into a proton and emits a electron and a electron-antineutrino. This decay increases the atomic number.
	$$n\rightarrow p+e^-+\bar{v}_e$$
	Another form of decay is the Electron Capture (EC) or $\epsilon-$decay. This one is similar to the $\beta+$decay since it also transforms a proton into a neutron. The difference being, that here the proton captures a electron to transform. The emitted particle is a electron-neutrino.
	$$p + e^- \rightarrow n + \bar{v}_e$$
	The captured electron is mostly from the K-shell while the resulting hole in the shell is filled by electrons from the L-shell. The remaining energy is either emitted through a X-ray photon or a Auger-electron. An Auger-electron is an electron that got the energy of an electron filling the vacancy left by electron in a lower state. The Auger-electron is therefore ejected. \\
	Decays are often accompanied by a $\gamma-$decays. When a decay occurs the daughter nucleus is mostly left in an exited state. It then decays into the ground state emitting $\gamma$-rays.\\
	Another Process similar to the $\gamma$-decay is the internal conversion (IC). Here the energy of a decay into a lower state is transmitted without radiation. That means no real photon is created to transport the energy. The energy is directly absorbed by another electron from the shell and ejected.	
	\subsection{Interaction between Matter and $\gamma-$Photons}	
 	When $\gamma-$photons and matter interact this happens mostly in 3 different ways depending on the atomic number of the atoms in the matter, as well as the Energy $E_\gamma$ of the photons.
 	\begin{enumerate}
 		\item Photoelectric effect:\\
 		The photoelectric effect happens when a photon is absorbed by an electron inside the matter. The energy carried by the photon is turned into kinetic energy and frees the electron. The vacancy is filled by electrons from higher shells and the energy is emitted by an Auger-electron or X-ray.\\
 		This effect appears mostly by $E_\gamma<200$\,keV and an atomic number around 50.
 		\item Compton scattering:\\
 		Unlike the photoelectric effect the photons are not absorbed by the electrons in the matter. They give up a part of their energy and scatter at the electron.\\
 		The Compton scattering occurs by energies in the range of $200$\,keV$<E_\gamma<5$\,MeV and a atomic number similar to the photoelectric effect.
 		\item Pair Production:\\
 		Pair production is an effect that appears by an energy $E_\gamma$ over the critical one of $1.022$\,MeV. When a $\gamma$-quantum gets into the electromagnetic field of a nucleus or electron it can be converted into an electron positron pair. 
 		$$\gamma \rightarrow e^- + e^+$$
 		To create this pair the energy of $1.022$\,MeV is needed this is also the reason the pair production can't happen if the photon has less energy. The remaining energy is given as kinetic energy to the electron and positron. The positron annihilates with an electron shortly after it's creation into two $\gamma$-rays with each half $0.511$\,MeV.
 	\end{enumerate}



\subsection{Radioaktiver Zerfall}
Der beim radioaktiven Zerfall verwandelt sich ein Instabiler Kern in einen leichteren unter Emission von Teilchen. Es existieren drei verschiedene Arten von radioaktiven Zerfall: 
\begin{itemize}
 \item[$\alpha$] {Bei dieser Zerfallsart stößt der Kern einen Heliumkern (ohne Elektronen) aus. Die Veränderung folgt diesem Schema:
$_Z^AX \rightarrow _{Z-4}^{A-2}Y + _4^2He^{2+}$ 
}
 \item[$\beta^-$]{Bei dieser Zerfallsart zerfällt ein } 
 \item[$\beta^+$]{}
\end{itemize}
 

	\section{Conduction of the experiment}
After the entrance exam the distances on the lid of the Dewar with the SQUID probe and the distance between the top of the Dewar and the position of the sample to later be able to compute the distance between the sample and the sensor. All distances were measured thee times to reduce the measurement inaccuracy. After that was finished, the Dewar was filled with the liquid nitrogen and the SQUID probe was placed inside to cool it down. While the sensor is cooling, the loop of the resistor measurements was measured from different angles because it is quite asymmetric. \par
Now, after approx. 15 minutes, the VCA and VCO settings in the control software of the SQUID were set as a calibration. They were modified so the SQUID signal has, as seen in figure \ref{cali_squid}, the characteristic differences from the usual sine function at the maxima and minima of the triangular reference voltage are as visible as possible.

\begin{figure}[ht]
\includegraphics[scale=0.5]{Bild/Eichung}
\centering
\caption[Picture of the calibration of the SQUID]{\small The figure shows the SQUID signal after being calibrated for the measurement. }
\label{cali_squid}
\end{figure}
\par
Now, after measuring the battery voltages, the measurements for the resistors were conducted. Starting with the smallest, four measurements of every resistor for each used motor speed were made. For the first resistor, the speed settings 10, 5 and 2 were used. During the measurements of the second resistor, it became clear that measurements of with the speed of 2 are not viable due to an increase in background interference. For the other resistors, only the speeds 10 and 5 were used and, also due to the increased background instabilities, only thee measurements each were made. After finishing the resistor measurements, the rotational speed of the motor settings 10 and 5 was measured over multiple rotations.\par
Now five different other samples were measured, each at a speed set to 10. The samples can be seen in figure \ref{samples} First a iron splinter, which worked well. Second a gold plate was tried, which did unfortunately not seem to have any measurable dipole moment. After taking two measurements, the signal suddenly disappeared and it seemed like, nothing was inside of the SQUID apparatus.Because there was a signal, the measurement should be retried later. The third sample was a magnet splinter, which also worked well.
After that, the gold sample was retried, and still no signal was measurable. Now, a stone was measured. After this, the gold sample was retried one last time, but it still showed no signal at all. As a last sample, a magnet was measured. It was chosen as the last sample, because it influences the detector so strong that for the rest of the day no other measurements can be made. \par
As the last measurement, the resistors were measured with the multimeter.

\begin{figure}[ht]
	\includegraphics[scale=0.1,angle=0]{Bild/samples}
	\centering
	\caption[Picture of the other samples]{\small The picture shows the other five samples used during the experiment. In the quadratic arranged samples, the top right one is the iron splinter, the top left is the magnet splinter. The bottom left one is the stone, the bottom right one is the magnet. To the right of the other samples, the gold sample is placed.}
	\label{samples}
\end{figure}
	\section{Auswertung}
\subsection{Extrapolation zur Ausschließung des Coherence Narrowing Effektes}
Die für verschiedene Temperaturen kann nun mit der Gleichung \ref{Coherence} der Druck bestimmt werden. Der Fehler des Drucks $\sigma_p$ wird mittels Gaußscher Fehlerfortpflanzung bestimmt:
\begin{equation}
\sigma_p = \frac{\partial p}{\partial T} \sigma_T
\end{equation}
Der Fehler für die Temperatur wurde auf $0.5\,$K geschätzt.	
	\section{Analyse der Bandlückenenergiebestimmung}
\subsection{Verfahren}
Das allgemeine Verfahren zur Bestimmung der Bandlückenenergie ist wie folgt:
Zum Beginn wurden zuerst die Messreihen vom Untergrund bereinigt und auf die Lampenenergie normiert. Hierzu wurde die Formel \ref{cleaner} verwendet. 
\begin{equation}
\label{cleaner}
\text{Trans}_{\text{Real}} = \frac{\text{\text{Trans }- \text{Untergrund}}}{\text{Lampe}_{\text{Trans}}} \qquad 
\text{Absorp}_{\text{Real}} = \frac{\text{\text{Absorp}- \text{Untergrund}}}{\text{Lampe}_{\text{Absorp}}}
\end{equation}
Nach dieser Korrektur wurden an den Stellen, wo Absorption und Transmission Gleichwahrscheinlich sind, Geraden angelegt und diese mit der Horizontalen des jeweiligen Maximum oder Minimum geschnitten. Bei der Transmission wurde das Maximum zum anlegen der Horizontalen verwendet, bei der Absorption (Widerstand der Probe) wurde die Horizontale an das Minimum angelegt. Dies kann man in Abbildung \ref{anleitung_fit} sehen. Aus diesen Punkten erhielt man die obere und untere Grenze für die Elektronen und Lochbildung. Der Wert für die Bandlückenenergie ist der Mittelwert dieser beiden Grenzen.
\begin{figure}[h]
	\centering
	\label{anleitung_fit}
	\includegraphics[scale=0.7]{Bilder/bsp_fit}
	\caption[Referenz zur Geraden Anpassung]{\small Beispiel zu den angepassten Geraden für die Bestimmung der Bandlückenenergie. Entnommen aus der Versuchsanleitung \cite{anleitung}}
\end{figure}

\subsection{Berechnung der Bandlückenenergie für Silizium}
Der erste Schritt bei der Normierung ist, die Messung der Lampe und die Silizium Messung auf die selbe Länge und Winkeländerung zu Synchronisieren. Die aufgezeichneten Messungen hatten leider beide obenerwähnten Änderungen, was korrigiert wurde. Es wurde begonnen, die ersten 18 zu verwerfen da hier keine Winkeländerung stattfindet. Als nun die Winkel zum Start der Daten nahezu übereinstimmten, wurde deutlich dass die Winkeländerungsrate der beiden Messungen unterschiedlich ist. Hierzu wurden bei einem Winkel in der Nähe von $0^{\circ}$ einige Datenpunkte ignoriert. Damit war die Anpassung abgeschlossen. Das Ergebnis der Anpassung ist in Abbildung \ref{anpassung_si_1} sichtbar. 

\begin{figure}[h]
	\centering
	\label{anpassung_si_1}
	\includegraphics[scale=0.5]{Bilder/korrektur_channels}
	\caption[Korrigierter Lampendatensatz]{\small Auftragung der gemessenen Winkel für die Lampenmessung und erste Silizium Messung nach der Korrektur. Beide stimmen in den relevanten Bereichen bei ca. $30^{\circ}-40^{\circ}$ weitgehend überein.}
\end{figure}

Es wurden mehrere Untergrundmessungen durchgeführt, wobei die ohne Gitter als am repräsentativsten gewählt wurde. Diese ist sichtbar in Abbildung \ref{untergrund} Da die Länge von der Untergrundmessung viel kürzer als die der eigentlichen Messungen, wurde bei dieser jeweils der Mittelwert als Repräsentation verwendet. 

\begin{figure}[h]
	\centering
	\label{untergrund}
	\includegraphics[scale=0.5]{Bilder/Hintergrund}
	\caption{\small Auftragung von gemessener Spannung der  Untergrundmessung gegen die Winkel.}
\end{figure}

Das Ergebnis der Normalisierung ist in der Abbildung \ref{normalized} dargestellt.

\begin{figure}[h]
	\centering
	\label{normalized}
	\includegraphics[scale=0.5]{Bilder/normalized}
	\caption{\small Auftragung von Intensität der normalisierten Datenreihen gegen die  Winkel.}
\end{figure}

Anschließend wurden die oben erwähnten Geraden angepasst. Dies wurde bei beiden Stellen der Messung durchgeführt. Anschließend wurden die Schnittpunkte der Geraden ($y = a\cdot +b$) mit den jeweiligen Horizontalen mittels der Formel \ref{schnitt} bestimmt. 
\begin{equation}
\label{schnitt}
x = \frac{x-b}{a}
\end{equation}
In Abbildungen \ref{si_1_l} und \ref{si_1_r} sind die angepassten Geraden, die horizontalen und deren Schnittpunkte dargestellt.
Die Parameter der Geraden, Horizontalen und die Winkel, bei welchen sie sich schneiden können aus der Tabelle \ref{ergebnis_si1} entnommen werden. Die so bestimmten Winkel der Schnittpunkte  wurden mit der Energie Datenreihe verglichen. Leider konnten sie nicht direkt abgelesen werden wegen der endlichen Genauigkeit der Aufgezeichneten Daten. Zunächst wurde versucht, eine Exponentialfunktion an die Daten anzupassen, was fehlschlug. Letztendlich wurde eine gerade an den Bereich angepasst, in welchem sich die Messwerte befinden. Diese sind in Abbildungen \ref{si_1_l_en} beispielhaft aufgetragen.

\begin{table}
	\centering
	\caption[Parameter erste Silizium Messung]{Parameter der angepassten Geraden, Horizontalen, Winkel der Schnittpunkte und die Energien zu den Schnittwinkeln der ersten Silizium Messung.}
	\label{ergebnis_si1}
	\begin{tabular}{lllll}
		\toprule
	Mess- &   Steigung und Offset  & Position der  & Winkel beim  & Energien der  \\
	reihe & der Geraden & Horizontalen & Schnittpunkt& Schnittpunkte \\
	
	\midrule
		1. Absorbtion   &  [0.0273+/-0.0008, 1.261+/-0.033] &                 0.0281946 &                                -45.2+/-1.7 &               -1.06+/-0.04 \\
		1. Transmission &  [-0.0494+/-0.0026, -1.86+/-0.10] &                  0.190734 &                                -41.5+/-3.0 &               -1.13+/-0.06 \\
		2. Absorbtion   &   [-0.0370+/-0.0015, 1.73+/-0.06] &                 0.0309053 &                                 46.0+/-2.5 &                1.04+/-0.05 \\
		2. Transmission &   [0.0431+/-0.0026, -1.66+/-0.11] &                  0.204556 &                                     43+/-4 &                1.09+/-0.07 \\
		\bottomrule
	\end{tabular}
\end{table}

\begin{figure}[h]
	\centering
	\label{si_1_l}
	\includegraphics[scale=0.5]{Bilder/si_1_l}
	\caption[Geraden Anpassungen erste Silizium Messung links]{\small Auftragung von Intensität der normalisierten Datenreihen gegen die  Winkel in der Nähe der Stelle Gleichwahrscheinlicher Absorption und Transmission bei Winkeln kleiner als $0^\circ$. Es sind zusätzlich die angepassten Geraden zur Absorption und Transmission eingezeichnet. Die Horizontalen wurden durch die Maxima der dahinterliegenden Datenpunkte bestimmt. Die Schnittpunkte der Geraden mit ihren jeweiligen Horizontalen sind auch eingezeichnet.}
\end{figure}

\begin{figure}[h]
	\centering
	\label{si_1_r}
	\includegraphics[scale=0.5]{Bilder/si_1_r}
	\caption[Geraden Anpassungen erste Silizium Messung rechts]{\small Auftragung von Intensität der normalisierten Datenreihen gegen die  Winkel in der Nähe der Stelle Gleichwahrscheinlicher Absorption und Transmission bei Winkeln größer als $0^\circ$. Es sind zusätzlich die angepassten Geraden zur Absorption und Transmission eingezeichnet. Die Horizontalen wurden durch die Maxima der dahinterliegenden Datenpunkte bestimmt. Die Schnittpunkte der Geraden mit ihren jeweiligen Horizontalen sind auch eingezeichnet.}
\end{figure}

\begin{figure}[h]
	\centering
	\label{si_1_l_en}
	\includegraphics[scale=0.5]{Bilder/si_1_l_energie}
	\caption[Beispiel Energiebestimmung]{\small Auftragung der Energie gegen den Winkel. Die angepasste Gerade und die beiden Schnittpunkte sind auch eingetragen.}
\end{figure}
 Für die zweite Silizium Messung wurde analog vorgegangen, die Bilder zu dieser sind im Anhang zu finden. Die Ergebnisse zur Auswertung sind in Tabelle \ref{ergebnis_si2} zu finden. 
 
 \begin{table}
 	\centering
 	\caption[Parameter der zweiten Silizium Messung]{Parameter der angepassten Geraden, Horizontalen, Winkel der Schnittpunkte und die Energien zu den Schnittwinkeln der zweiten Silizium Messung.}
 	\label{ergebnis_si2}
 	\begin{tabular}{lllll}
 		\toprule
 		Mess- &   Steigung und Offset  & Position der  & Winkel beim  & Energien der  \\
 		reihe & der Geraden & Horizontalen & Schnittpunkt& Schnittpunkte \\
 		
 		\midrule
 		1. Absorbtion   &    [0.0327+/-0.0013, 1.49+/-0.05] &                 0.0274841 &                                -44.8+/-2.3 &               -1.11+/-0.05 \\
 		1. Transmission &  [-0.0398+/-0.0023, -1.49+/-0.09] &                  0.188553 &                                -42.2+/-3.3 &               -1.17+/-0.07 \\
 		2. Absorbtion   &   [-0.0418+/-0.0017, 1.94+/-0.07] &                 0.0300767 &                                 45.7+/-2.5 &                1.01+/-0.04 \\
 		2. Transmission &   [0.0486+/-0.0021, -1.88+/-0.09] &                  0.205715 &                                 42.9+/-2.6 &                1.06+/-0.04 \\
 		\bottomrule
 	\end{tabular}
 	
 	
 \end{table}
	
	\include{Kapitel/AnalyseV1}
	\section{Auswertung von Haynes \& Shockley}
\subsection{Messung bei Konstanter Abstand}
	Als erstes wurden die einzelnen gemessenen Datenreihen geplottet und mit einer Gaußschen Normalverteilung gefittet. Die verwendete Form ist in Gleichung \ref{Gaus} zu finden.
	\begin{equation}
		f(x) = A \frac{1}{\sqrt{2\pi \sigma^2}} 	\exp\left(-\frac{1}{2}\frac{(x-x_c)^2}{\sigma^2}\right)+h
		\label{Gaus}
	\end{equation}
	Hierfür wurde das Python Paket \verb|scipy.optimize| mit der Funktion \verb|curve_fit| verwendet. Die Bilder der Messungen sind im Anhang.
	Die Erhaltenen Parameter wurden ohne Offset $h$ abgebildet in Abbildung \ref{SpannungGaus} dargestellt.	Wenn man die Gaußkurve mit der Differentialgleichung \ref{DifferentialGL} welche die Bewegung der Elektronenwolken in Halbleiter beschriebt, vergleicht
	\begin{equation}
		c(t,x)=C\exp\left(-\frac{t}{\tau_n}\right)\cdot\exp\left(-\frac{(x-\mu_nEt	)^2}{4D_nt}\right)
		\label{DifferentialGL}
	\end{equation}
	erhält man folgende Gleichungen für die einzelnen Parameter der Gaußkurve \ref{Gaus}:
	\begin{equation}
		x_c(t)=\mu_nEt \qquad A(t)=\exp\left(-\frac{t}{\tau_n}\right) \qquad \sigma(t)^2=2D_nt
	\end{equation}
	Hierbei sind $\mu_n$ die Beweglichkeit der Elektronenwolken, $\tau_n$ ihre Lebenszeit und $D_n$ die Diffusionskonstante.\par
	\FloatBarrier
	\begin{figure}[ht]
		\includegraphics[scale=0.45]{Bild/V2Abstand1}
		\centering
		\caption[Darstellung der Gaußkurven bei konst. Abstand]{Gemeinsame Darstellung der Gaußkurven bei konstanten Abstand ohne Offset nebeneinander.}
		\label{SpannungGaus}
	\end{figure}
	\FloatBarrier
	Als erstes wird versucht die Beweglichkeit $\mu_n$ zu bestimmen. Hierfür wurde die Inverse Spannung $\frac{1}{U(t)}$ gegen die Zeit geplottet und mit einer gewichteten linearen Regression gefittet. Als Fehler wurden die Fehler der Zeit auf der y-Achse verwendet. Siehe Abbildung \ref{SpannungBew}. Das multiplizieren der Steigung $m$ mit der Länge des Halbleiters $l=3$\,cm so wie dem Abstand zwischen Laser und Nadel $d=3.6$\,mm ergibt nun die gesuchte Beweglichkeit.
	\begin{equation}
		\mu_1=ldm
	\end{equation}
	Der Fehler ergibt sich durch Gaußsche Fehlerfortpflanzung mit der Gleichung \ref{FFS1} wo bei die Fehler auf die Abstände beide auf $0.1$\,mm geschätzt wurden.
	\begin{equation}
		\sigma_{\mu_1}=\sqrt{\left(dm\sigma_l\right)^2+\left(lm\sigma_d\right)^2+\left(dl\sigma_m\right)^2}
		\label{FFS1}
	\end{equation}
	Dies ergab einen Wert von $\mu_1=\left(3.90 \pm 0.07\right) \times 10^{3}\,\frac{\text{cm}^2}{\text{Vs}}$
	Nun wurde die Lebenszeit über die Amplitude der Gaußfits bestimmt. Hierfür wurde als Amplitude $A \frac{1}{\sqrt{2\pi \sigma^2}}$ verwendet und dies gegen die Zeit geplottet. Die Datenpunkte wurden dann wie in Abbildung \ref{SpannungTau} zu sehen exponentiell gefittet mit der Form \ref{Expotentialform}.
	\begin{equation}
		f(x)=C\exp\left(-\frac{t}{\tau}\right)+h
		\label{Expotentialform}
	\end{equation}
	Für den Parameter und damit die Lebenszeit ergab sich ein Wert von $\tau_1=\left(1.28 \pm 0.08\right) \times 10^{-6}\,s$
	Nun wird die Diffusionskonstante $D_1$ bestimmt, indem $\sigma^2$ gegen die Zeit dargestellt wird und eine linearer Fit an diese Werte angepasst wird (siehe Abbildung \ref{SpannungD}). Da die Fehler auf das Sigma durch das Quadrieren relativ groß wurden wurde ein gewichteter fit mit Fehlern auf $\sigma^2$ durchgeführt. Das halbieren der Steigung des Fits ergibt für die Diffusionskonstante  $D_1=\left(3.4 \pm 0.5\right) \times 10^{-5}
	\,\frac{\text{cm}^2}{\text{s}}$.
	Die Werte sind noch einmal gemeinsam in Tabelle \ref{MesswerteV2} dargestellt, zusammen mit dem erwarteten Literaturwert.
	\FloatBarrier
	\begin{figure}[ht]
		\includegraphics[scale=0.43]{Bild/V2Abstand3}
		\centering
		\caption[Exponentieller Fit der Amplituden bei konstantem Abstand]{\small Exponentieller Fit der Amplituden bei Konstantem Abstand. Fehler wurden nicht eingezeichnet, da sie nicht sinnvoll zu erkennen waren.}
		\label{SpannungTau}
	\end{figure}
	\begin{figure}[ht]
		\includegraphics[scale=0.43]{Bild/V2Abstand4}
		\centering
		\caption[Fit zur Bestimmung der Diffusion bei konst. Abstand.]{\small Datenpunkte von $\sigma^2$ gegen die Zeit mit Fehlern. In orange der gerade Fit zur Bestimmung der Diffusionskonstante.}
		\label{SpannungD}
	\end{figure}
	\begin{figure}[ht]
		\includegraphics[scale=0.5]{Bild/V2Abstand2}
		\centering
		\caption[Linearer Fit zur Bestimmung der Beweglichkeit bei konst. Abstand]{Linearer Fit zur Bestimmung der Beweglichkeit bei konst. Abstand. Fehler wurden nicht beigefügt, da diese sich nicht sinnvoll darstellen ließen.}
		\label{SpannungBew}
	\end{figure}
	\FloatBarrier
	\newpage
	\subsection{Messung bei Konstante Spannung}
	Für die Messung mit Konstanter Spannung werden wie zuvor die Daten mit der Gleichung \ref{Gaus} gefittet und sind in Abbildung \ref{Abstand1} ohne Offset dargestellt.\par
	\FloatBarrier
	\begin{figure}[ht]
		\includegraphics[scale=0.5]{Bild/V2Spannung1}
		\centering
		\caption[Darstellung der Gaußkurven bei konst. Spannung]{\small Gemeinsame Darstellung der Gaußkurven bei konstanter Spannung ohne Offset nebeneinander.}
		\label{Abstand1}
	\end{figure}
	\FloatBarrier
	Als nächstes wurde die Beweglichkeit $\mu_2$ bestimmt indem der gemessene Abstand zwischen Nadel und Laser gegen die Zeit gefittet, welche die Elektronenwolke benötigte. Hierbei wurden die Fehler auf die Zeit mitberücksichtigt. Die Steigung $m$ erhält man mit Gleichung \ref{AbstandBesch}.
	\begin{equation}
		\mu_2=\frac{1}{mE} \qquad \qquad \sigma_{mu_2}=\sqrt{\left(\frac{\sigma_{m}}{m^2E}\right)^2+\left(\frac{\sigma_{E}}{mE^2}\right)}
		\label{AbstandBesch}
	\end{equation}
	$E$ kann hier über die angelegte Spannung $U=(48\pm0.5)\,$V und die Länge $l=3\pm0.1\,$cm mit Gleichung \ref{E} bestimmt werden.
	\begin{equation}
		E=\frac{U}{l} \qquad \qquad \sigma_E=\sqrt{\left(\frac{\sigma_U}{l}\right)^2+\left(\frac{U\sigma_l}{l^2}\right)^2}
		\label{E}
	\end{equation}
	Damit ergibt sich für $\mu_2=(3136 \pm 32)\,\frac{\text{cm}^2}{Vs}$.\par
	\begin{figure}[ht]
		\includegraphics[scale=0.5]{Bild/V2Spannung2}
		\centering
		\caption[Linearer Fit zur Bestimmung der Beweglichkeit bei konst. Spannung]{\small Linearer Fit zur Bestimmung der Beweglichkeit bei konst. Spannung. Fehler wurden nicht beigefügt, da diese sich nicht sinnvoll darstellen ließen.}
		\label{Abstand}
	\end{figure}
	Danach wird wie bei dem Konstanter Spannung die Lebenszeit wie zuvor über die Amplitude bestimmt. Hierzu wird Gleichung \ref{Expotentialform} benutzt wie in Abbildung \ref{AbstandTau} dargestellt. Für die Lebenszeit ergibt sich dadurch ein Wert von $\tau_2= \left(5.9 \pm 1.0\right) \times 10^{-6}\,s$.\par
	\begin{figure}[ht]
		\includegraphics[scale=0.5]{Bild/V2Spannung3}
		\centering
		\caption[Exponentieller Fit der Amplituden bei konstanter Spannung]{\small Exponentieller Fit der Amplituden bei Konstanter Spannung. Fehler wurden nicht eingezeichnet, da sie nicht sinnvoll zu erkennen waren.}
		\label{AbstandTau}
	\end{figure} 
	Für die Diffusionskonstante wird wieder der Parameter $\sigma^2$ mit Fehlern gegen die Zeit aufgetragen und mit einem gewichteten Fit angepasst. Beide sind in Abbildung \ref{AbstandD}. Es ergibt sich durch halbieren der Steigung ein Wert von $D_2=\left(2.8 \pm 0.7\right) \times 10^{-6}$. Die Werte sind noch einmal gemeinsam in Tabelle \ref{MesswerteV2} dargestellt zusammen mit dem erwarteten Literaturwert.
	\begin{figure}[ht]
		\includegraphics[scale=0.5]{Bild/V2Spannung4}
		\centering
		\caption[Fit zur Bestimmung der Diffusion bei konst. Spannung.]{\small Datenpunkte von $\sigma^2$ gegen die Zeit mit Fehlern. In orange der gerade Fit zur Bestimmung der Diffusionskonstante.}
		\label{AbstandD}
	\end{figure}
	\begin{table}[ht]
		\begin{Dtabular}[1.1]{|c|c|c|c|}
			\hline
			&Beweglichkeit $\mu$ [$\frac{\text{cm}^2}{\text{Vs}}$]&Lebenszeit $\tau$ [$\mu$s]& Diffusion $D$ [$\frac{\text{cm}^2}{\text{s}}$]\\
			\hline
			Messung bei konst. Abstand&$\left(3900 \pm 70\right)$&$\left(1.28 \pm 0.08\right)$&$\left(3.4 \pm 0.5\right) \times 10^{-5}$\\
			\hline
			Messung bei konst. Spannung&$(3136 \pm 32)$&$\left(5.9 \pm 1.0\right)$&$\left(2.8 \pm 0.7\right) \times 10^{-6}$\\
			\hline
			Literaturwert&$3900$&$(45\pm2)$&$101$\\
			\hline
		\end{Dtabular}
		\centering
		\caption{Messwerte von Versuchsteil 2 mit Literaturwerten\cite{anleitung}}
		\label{MesswerteV2}
	\end{table}
	\section{Auswertung von Halbleiter Experiment}
Im Versuchsteil 3 wurden die Energiespektren von $^{241}$Am und $^{57}$Co mit zwei unterschiedlichen Detektoren aufgenommen. Für die Auswertung wurden hier der $122.06\,$keV so wie der $136.47\,$keV Photopeak von Cobalt so wie der $59.5\,$keV Peak von Americium mit einer Gaußkurve wie in Gleichung \ref{Gaus} gefittet. Die Peaks so wie die dazugehörigen Anpassungen für $^{57}$Co sind in Abbildung \ref{CoS} und \ref{CoC} zu sehen. Die für Americium sind in Abbildung \ref{AmS} und \ref{AmC} zu finden. Die gesamten Spektren sind im Anhang.\par
\begin{figure}[ht]
	\includegraphics[scale=0.5]{Bild/CS.png}
	\centering
	\caption[V3 $122.06\,$keV und $136.47\,$keV Peaks mit Silizium Detektor]{Gaußfit für einen denn Datenbereich der $122.06\,$keV und $136.47\,$keV Photopeaks mit dem Silizium Detektor.}
	\label{CoS}
\end{figure}
\begin{figure}[ht]
	\includegraphics[scale=0.5]{Bild/CC.png}
	\centering
	\caption[V3 $122.06\,$keV und $136.47\,$keV Peaks mit CdTe Detektor]{Gaußfit für einen denn Datenbereich der $122.06\,$keV und $136.47\,$keV Photopeaks mit dem CdTe Detektor.}
	\label{CoC}
\end{figure}
\begin{figure}[ht]
	\includegraphics[scale=0.5]{Bild/AS.png}
	\centering
	\caption[V3 $59.5$\,keV Peaks mit Silizium Detektor]{Gaußfit für einen denn Datenbereich der $59.5$\,keV Photopeaks mit dem Silizium Detektor.}
	\label{AmS}
\end{figure}
\begin{figure}[ht]
	\includegraphics[scale=0.5]{Bild/AC.png}
	\centering
	\caption[V3 $59.5$\,keV Peaks mit CdTe Detektor]{Gaußfit für einen denn Datenbereich der $59.5$\,keV Photopeaks mit dem CdTe Detektor.}
	\label{AmC}
\end{figure}
\FloatBarrier
Die Position der Peaks welche die Kanalnummer des MCA's ist kann nun mit Hilfe der bekannten Energien geeicht werden, indem man separat für beide Detektoren Energie gegen Position aufträgt und die Werte dann linear anpasst. Hierbei wurde ein gewichteter Fit verwendet welche die Fehler auf die Position der Peaks berücksichtigte. Die Werte so wie die angepasste Kurve sind in Abbildung \ref{Eichung} zu sehen. Die erhaltenen Gleichungen für denn Zusammenhang zwischen Kanal und Energie sind:\par
\begin{equation*}
	f(E)_{Kanal} = (5.3988\pm 0.0010)\,\frac{1}{\text{keV}}E+(4.33\pm 0.10)
\end{equation*}
\begin{equation*}
f(E)_{Kanal} = (5.128\pm 0.008)\,\frac{1}{\text{keV}}E+(6.7\pm 0.8)
\end{equation*}
\begin{figure}[ht]
	\includegraphics[scale=0.5]{Bild/Eichung.png}
	\centering
	\caption[Eichung für beide Detektoren]{Geraden zur Eichung der beiden Detektoren. In blau die Eichung für den Silizium Detektor und in grün für den CdTe Detektor. Fehler sind nicht mit eingezeichnet da sie zu klein waren um sie sinnvoll darzustellen.}
	\label{Eichung}
\end{figure}
Nun wird mit das Absorptionsverhältnis der beiden Detektoren bei unterschiedlichen Energien bestimmt. Dafür müssen die aus den Fit bestimmten Amplituden erst einmal aktive Fläche normiert werden und dann durcheinander geteilt wie in Gleichung \ref{Abs} dargestellt:
\begin{eqnarray}
\frac{Abs_{\text{Si}}}{Abs_{\text{CdTe}}}=\frac{\nicefrac{A_{\text{Si}}}{a_{\text{Si}}}}{\nicefrac{A_{\text{CdTe}}}{a_{\text{CdTe}}}}
\label{Abs}
\end{eqnarray}
Die Fläche des CdTe Detektors beträgt $a_\text{CdTe}=23\,\text{mm}^2$ und die des Si Detektors $a_{\text{Si}}=100\,\text{mm}^2$. Die erhaltenen Werte sind zusammen mit den Literaturwerten aus der Anleitung\cite{anleitung} in Tabelle \ref{MesswerteV3_1} zu finden.
Als letztes wird die Relative Energieauflösung der einzelnen Peaks bestimmt. Zur Berechnung wird die Halbwertsbreite verwendet wodurch sich folgende Gleichung ergibt:
\begin{equation}
	RER(E)=\frac{FWHM}{E}\approx\frac{2.35\sigma(E)}{E}
\end{equation}
Die erhaltenen Werte sind in Tabelle \ref{MesswerteV3_2} zu finden.
\begin{table}[ht]
	\begin{Dtabular}[1.1]{|c|c|c|}
		\hline
		Photopeak Energie [keV]&Berechnete Werte [\%] &Literaturwerte [\%]\\
		\hline
		$59.5$&$1.28 \pm 0.07$&$1,40$\\
		\hline
		$122.06$&$0.58 \pm 0.05$&$1,83$\\
		\hline
		$136.47$&$0.45 \pm 0.06$&$2,00$\\
		\hline
	\end{Dtabular}
	\centering
	\caption{Absorptionsverhältnis von Versuchsteil 3 mit Literaturwerten\cite{anleitung}}
	\label{MesswerteV3_1}
\end{table}
\begin{table}[ht]
	\begin{Dtabular}[1.1]{|c|c|c|c|}
		\hline
		Detektor und Energie&Energieauflösung\\
		\hline
		Silizium $59.5$&$0.449 \pm 0.007$\\
		\hline
		Silizium $122.06$&$0.236 \pm 0.005$\\
		\hline
		Silizium $136.47$&$0.200 \pm 0.010$\\
		\hline
		CdTe $59.5$&$0.600 \pm 0.012$\\
		\hline
		CdTe $122.06$&$0.285 \pm 0.007$\\
		\hline
		CdTe $136.47$&$0.272 \pm 0.010$\\
		\hline
	\end{Dtabular}
	\centering
	\caption{Energieauflösung von Versuchsteil 2 mit Literaturwerten\cite{anleitung}}
	\label{MesswerteV3_2}
\end{table}	
	\section{Zusammenfassung}
Wenn man die gemessenen Werte mit denen des Literatur Wertes von $119\,$ns mit der Formel \ref{Vergleich} vergleicht erhält man die in Tabelle \ref{VglTable} beschriebenen Werte. \\
\begin{equation}
t=\frac{\left\|a-b\right|}{\Delta a}
\label{Vergleich}
\end{equation}


%\begin{table}
%	\label{VglTable}
%	\begin{Dtabular}[1.1]{|c|c|c|}
%		\hline
%		Messreihe&Lebensdauer $\tau$[ns]&Vergleichswert\\
%		\hline
%		Abkühlen 1 bei $0^\circ$&$122.4\pm2.2)$&$1.5$\\
%		\hline
%		Abkühlen 1 bei $90^\circ$&$122.2\pm2.0$&$1.6$\\
%		\hline
%		Aufwärmen bei $0^\circ$&$102.8\pm1.3$&$1.4$\\
%		\hline
%		Abkühlen 2 bei $0^\circ$&$116.6\pm1.7$&$0.5$\\
%		\hline
%		Abkühlen 2 bei $90^\circ$&$118.1\pm1.8$&$12.5$\\
%		\hline
%	\end{Dtabular}
%\end{table}

\begin{center}
	\begin{table}[h]
		\centering
		\begin{tabular}{|c|c|c|}
			\hline
			Messreihe&Lebensdauer $\tau$[ns]&Vergleichswert\\
			\hline
			Abkühlen 1 bei $0^\circ$&$122.4\pm2.2)$&$1.5$\\
			\hline
			Abkühlen 1 bei $90^\circ$&$122.2\pm2.0$&$1.6$\\
			\hline
			Aufwärmen bei $0^\circ$&$102.8\pm1.3$&$1.4$\\
			\hline
			Abkühlen 2 bei $0^\circ$&$116.6\pm1.7$&$0.5$\\
			\hline
			Abkühlen 2 bei $90^\circ$&$118.1\pm1.8$&$12.5$\\
			\hline
		\end{tabular}
	\caption[Endergebnisse]{Vergleich der berechneten Lebensdauern mit dem Literaturwert}
	\label{VglTable}
	\end{table}
\end{center}




Man erkennt schnell, dass alle gemessenen Werte bis auf die Aufwärmmessung mit dem Literaturwert kompatibel sind. Die große Diskrepanz kann man dadurch erklären, dass für die Messung während des Aufwärmvorgangs nicht gewartet wurde bis die Temperatur des Thermometers mit der der Probe angeglichen hat. Dadurch ziehen wir einen großen systematischen Fehler mit welcher die Unverträglichkeit erklären könnte. \par 
Ein weiteres Problem ist sicher die geringe Anzahl an Messpunkten die wir in allen Messreihen hatten, wodurch sich natürlich unsere Ergebnisse verschlechtern. Bei einer erneuten Durchführung des Versuches wäre es daher Sinnvoll, deutlich weniger Zeit auf die Kalibrierung der Erdmagnetfeldkompensation zu verwenden und stattdessen längere Abkühlungsmessungen durchzuführen und bei der Erwärmungsmessung die Leistung des Kühlaggregates langsam zurückzufahren statt es auszuschalten.
	\section{Zusammenfassung und Diskussion zum Versuchsteil 2}
Im Versuchsteil 2 ging es darum die Elektronenwolken innerhalb eines Halbleiters zu untersuchen. Hierfür wurden ein $3$cm langer Germanium Block genutzt. Zur Messung wurde einmal der Abstand zwischen Laser und Nadel verändert und einmal die am Germanium angelegte Spannung.
Die berechneten Werte sind die Lebensdauer, Beweglichkeit so die Diffusion der Wolken und sind in Tabelle \ref{MesswerteV2} zu finden. Zum Vergleich der Werte mit dem Literaturwert wurde die Gleichung \ref{vgl} verwendet. Die Werte sind in Tabelle \ref{VGLV2} notiert.
\begin{equation}
	t=\frac{x_{Mess}-x_{Literatur}}{\sigma_{x_{Messung}}}
	\label{vgl}
\end{equation}
\FloatBarrier
\begin{table}[ht]
	\begin{Dtabular}[1.1]{|c|c|c|}
		\hline
		&t bei konst. Abstand $[\sigma]$&t bei konst. Spannung $[\sigma]$\\
		\hline
		$\mu$ &$0.0$&$23.875$\\
		\hline
		$\tau$ &$546.5$&$39.1$\\
		\hline
		$D$ &$202.0$&$144.3$\\
		\hline
	\end{Dtabular}
	\centering
	\caption[Vergleichswerte V2]{Vergleichswerte der Berechneten Werte mit dem Literaturwerten über Gleichung \ref{vgl}}
	\label{VGLV2}
\end{table}
Wenn man sich diese Werte anschaut passen bis auf den Wert bei $\mu$ mit konstantem Abstand keine der Werte überein. Die große Abweichung bei der Lebenszeit kann man damit erklären, dass die freien Elektronen tiefer in der Germanium Probe erzeugt werden. Gittereffekte naher der Oberfläche können daher zu ein Grund für die starke Verkürzung der Lebenszeit sein. Wenn man sich die Diffusionskonstanten anschaut fällt auf, dass diese nicht einmal in der selben Größenordnung sind. Hier könnte der Grund ein Fehler in der Auswertung sein, welcher jedoch nicht gefunden worden ist. Ein weiterer Grund für die großen Unterschiede könnte auch sein, dass man den Gaußfit mit einer Gerade hätte überlagern sollen. Da dies nicht getan wurde kann dies zu systematischen Fehlern geführt haben die, die Unterschiede hervorrufen. Jedoch sollte die hierbei entstandene Diskrepanz nicht den Unterschied von mehreren Größenordnungen verursacht haben.

	\section{Zusammenfassung und Diskussion zum Versuchsteil 3}
Im Versuchsteil 3 ging es um Halbleiterdetektoren. Hierbei wurde eine Silizium Diode und ein CdTe Kristall verwendet. Mit ihnen wurden die Energiespektren von $^ {57}$Co und $^{241}$Am aufgezeichnet. Mit diesen Spektren wurde eine Energieeichung durchgeführt und dann die Energieauflösung der einzelnen Peaks so, wie das Absorptionsverhältnis der beiden Detektoren bei den Unterschiedlichen Peaks bestimmt.\par
Wenn man sich die Eichungen anschaut scheinen die Messwerte sehr gut zueinander zu passen, es fällt jedoch auf, dass die geraden eine Abweichung der Steigung voneinander haben, was zeigt, dass eine individuelle Kalibrierung sinnvoll ist.\par

Wenn man sich die Werte für die Absorptionsverhältnisse anschaut (siehe Tabelle \ref{MesswerteV3_1}) fällt auf das es einen großen unterschied zwischen denen bei höheren Energien also den beiden $^{57}$Co Photopeaks der Wert geringer ist als bei der niedrigeren vom $59.5\,$keV Peak von Americium. Dies steht etwas im Widerspruch zu den Literaturwerten bei denen das Absorptionsverhältnis für höhere Energien größer ist. Gründe dafür können Abweichungen der Herstellerangaben sein oder ein großer Verlust an Signalstärke durch Ladungsrekombination verloren haben, dass sie dem ursprünglichen Peak nicht mehr zugeschrieben werden können. Auch wurde die Absorption in der Epoxid-Schicht so wie der Si02-Schicht der Silizium-Diode nicht mitberücksichtigt.\par
Bei Betrachtung der Energieauflösung in Tabelle \ref{MesswerteV3_2} fällt auf, dass die Auflösung des Silizium Detektors besser ist als die des CdTe Detektors. Auch scheint sich die Auflösung bei höheren Energien merklich zu verbessern. So scheinen sich die Photopeaks von Cobalt fast doppelt so gut auflösen wie beim $59.5$\,keV Peak von Americium.

	
	
	
	
	\section{Tabellen}
	\listoftables
	\section{Bilder}
	\listoffigures
	\section{Bibliograpy}
	\bibliographystyle{plain}
	\bibliography{Quellen}
	\addcontentsline{toc}{section}{Literatur}
	\section{Anhang}
	\begin{figure}[ht]
	H
\end{figure}
\end{document}