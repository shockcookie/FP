	\begin{figure}[ht]
		\includegraphics[scale=0.8]{Bild/BspLockIn}
		\centering
		\caption[Expected Signal for the Lock-In Method]{Derived absorption signal in red. Superposition of sinus and sawtooth waves with both 'Nulldurchgängen' aligned.\cite{anleitung}}
		\label{SägezahnBsp}
	\end{figure}
	\begin{table}[ht]
	\begin{Dtabular}[1.1]{|c|c|c|c|}
		\hline
		CSV File&Parameter $m$&Parameter $b$&$d_s$\\
		\hline
		\verb|lockin_2|&$0.01090\pm0.00011$&$-8.16\pm0.08$&$748\pm11 $\\
		\hline
		\verb|lockin_3|&$0.01091\pm0.00011$&$-3.74\pm0.04$&$1022\pm22$ \\
		\hline
		\verb|lockin_4|&$0.01093\pm0.00016$&$-11.16\pm0.16$&$518\pm8$\\
		\hline
		\verb|lockin_5|&$0.01090\pm0.00010$&$-5.65\pm0.06$&$945\pm17$\\
		\hline
		\verb|lockin_6|&$0.01092\pm0.00013$&$-10.32\pm0.12$&$375\pm6$\\
		\hline
		\verb|lockin_7|&$0.01094\pm0.00011$&$-4.10\pm0.04$&$472\pm7$\\
		\hline
		\verb|lockin_8|&$0.01090\pm0.00010$&$-5.14\pm0.05$&$355\pm5$\\
		\hline
		\verb|lockin_8|&$0.01091\pm0.00011$&$-3.87\pm0.04$&$355\pm5$\\
		\hline
	\end{Dtabular}
		\centering
		\caption[Parameter of Sawtooth]{Parameters and position of the 'Nulldurchgang' of the sawtooth fit. Here $m$ is the slope, $b$ the crossing of the y-axis and $d_s$ the 'Nulldurchgang'.}
		\label{SägezahnParameter}
	\end{table}
	\begin{table}[ht]
		\begin{Dtabular}[1.1]{|c|c|c|c|c|c|}
			\hline
			CSV File&$a$&$d_A$&$c$&$h$&Frequency[MHz]\\
			\hline
			\verb|lockin_2|&$16.3 \pm 0.6$&$335.3 \pm 0.5$&$18.0 \pm 0.5$&$0.0076 \pm 0.0035$&$19.2625\pm0.00005$\\
			\hline
			\verb|lockin_3|&$16.3 \pm 0.6$&$203.6 \pm 0.5$&$-18.8 \pm 0.5$&$0.0072 \pm 0.0035$&$19.2629\pm0.00005$\\
			\hline
			\verb|lockin_4|&$16.6 \pm 0.6$&$211.6 \pm 0.5$&$18.9 \pm 0.5$&$0.0082 \pm 0.0035$&$19.0935\pm0.00005$\\
			\hline
			\verb|lockin_5|&$15.7 \pm 0.6$&$1014.0 \pm 0.5$&$18.2 \pm 0.5$&$0.0086 \pm 0.0034$&$19.0041\pm0.00005$\\
			\hline
			\verb|lockin_6|&$16.7 \pm 0.5$&$814.9 \pm 0.5$&$17.0 \pm 0.5$&$0.0077 \pm 0.0034$&$19.1063\pm0.00005$\\
			\hline
			\verb|lockin_7|&$15.3 \pm 0.6$&$1134.8 \pm 0.5$&$17.1 \pm 0.5$&$0.0050 \pm 0.0035$&$19.2091\pm0.00005$\\
			\hline
			\verb|lockin_8|&$16.7 \pm 0.5$&$97.0 \pm 0.5$&$-18.5 \pm 0.5$&$0.0078 \pm 0.0032$&$19.2518\pm0.00005$\\
			\hline
			\verb|lockin_8|&$16.4 \pm 0.6$&$734.6 \pm 0.5$&$18.4 \pm 0.5$&$0.0072 \pm 0.0035$&$19.1344\pm0.00005$\\
			\hline
		\end{Dtabular}
	
		\centering
		\caption[Parameter of Derived Gaussian]{Parameters and position of the 'Nulldurchgang' of the derived Gaussian fit.}
		\label{GaussianTable}
	\end{table}
	\begin{figure}[ht]
		\includegraphics[scale=0.5]{Bild/LockIn2.png}
		\centering
		\caption[Plots and Fits of Lock-In Method 2]{\small The upper figure shows the data of the sawtooth in blue with the corresponding fit in orange. The absorption signal is in red. The lower one shows the part of the absorption signal which is of interest with the fit in blue. This figure is of the CSV file lockin 2.}
		\label{Lock2}
	\end{figure}
	\begin{figure}[ht]
		\includegraphics[scale=0.5]{Bild/LockIn3.png}
		\centering
		\caption[Plots and Fits of Lock-In Method 3]{\small The upper figure shows the data of the sawtooth in blue with the corresponding fit in orange. The absorption signal is in red. The lower one shows the part of the absorption signal which is of interest with the fit in blue. This figure is of the CSV file lockin 3.}
		\label{Lock3}
	\end{figure}
	\begin{figure}[ht]
		\includegraphics[scale=0.5]{Bild/LockIn4.png}
		\centering
		\caption[Plots and Fits of Lock-In Method 4]{\small The upper figure shows the data of the sawtooth in blue with the corresponding fit in orange. The absorption signal is in red. The lower one shows the part of the absorption signal which is of interest with the fit in blue. This figure is of the CSV file lockin 4.}
		\label{Lock4}
	\end{figure}
	\begin{figure}[ht]
		\includegraphics[scale=0.5]{Bild/LockIn5.png}
		\centering
		\caption[Plots and Fits of Lock-In Method 5]{\small The upper figure shows the data of the sawtooth in blue with the corresponding fit in orange. The absorption signal is in red. The lower one shows the part of the absorption signal which is of interest with the fit in blue. This figure is of the CSV file lockin 5.}
		\label{Lock5}
	\end{figure}
	\begin{figure}[ht]
		\includegraphics[scale=0.5]{Bild/LockIn6.png}
		\centering
		\caption[Plots and Fits of Lock-In Method 6]{\small The upper figure shows the data of the sawtooth in blue with the corresponding fit in orange. The absorption signal is in red. The lower one shows the part of the absorption signal which is of interest with the fit in blue. This figure is of the CSV file lockin 6.}
		\label{Lock6}
	\end{figure}
	\begin{figure}[ht]
		\includegraphics[scale=0.5]{Bild/LockIn7.png}
		\centering
		\caption[Plots and Fits of Lock-In Method 7]{\small The upper figure shows the data of the sawtooth in blue with the corresponding fit in orange. The absorption signal is in red. The lower one shows the part of the absorption signal which is of interest with the fit in blue. This figure is of the CSV file lockin 7.}
		\label{Lock7}
	\end{figure}
