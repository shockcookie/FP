%Pictures (ref) missing
\section{Conduct of the Experiment}
In the beginning of the experiment it had to be checked if the magnetic field inside the setup is homogeneously distributed. For this a Hall-Sensor was used with which the field strength can be measured. Here by the sensor was slowly put into the field and depending on place the strength of the field was recorded. After conforming the uniformity of the field, a position in the middle of the homogeneous part was chosen to place the probes into.\par
With this set the experiment was set up like in figure \ref{Versuchaufbau2} described. The measurement was started with Glycol at a depth of $2$\,cm and a constant magnetic field of $425\,$mT. With this set, the corresponding frequency of the nuclear magnetic resonance (NMR) oscillator was set by looking for the absorption peaks in the oscilloscope. After learning that it would be easier to find the rough position with a fixed frequency it became much easier to locate the peaks. The fine positioning was still done by modifying the frequency. By setting all peaks equidistant to one another the correct resonance frequency could be found. After this two underground samples were made, one without the H$_1$ probe inside the field and by setting wrong combinations of magnetic field and oscillation. The H$_1$ probe was used for all background as well as for the calibration between the change in frequency and the change in position on the oscilloscope. The reason for using H$_1$ is that it has the least amount of disturbances. It is so to speak our standard candle. With the position - frequency calibration done the actually measurements were started anew. For different combinations of magnetic field and frequency with equidistant absorption lines, the CSV files were taken. After doing this for H$_1$ and Glycol the Hall-Sensor broke and another one had to be used. This one had the problem that it was strongly influenced by the temperature. This showed by the slowly decrease in measured field strength. The first value measured around the depth of $2\,$cm was noted.\par 
After finishing this way the $^{19}$F probe the Lock in Method was used to decrease the background noise. After building the setup of figure \ref{Versuchsaufbau3} a suitable resonance frequency was chosen. Here a slow shifting of the absorption peaks was noted, most likely duo to the heating of the magnets. After waiting for them to be warmed up the shifting stopped and the first measurement with the Lock in Method could be started. Here a new calibration of position and frequency was made by shifting the positions of the differentiated absorption signal.\par
Duo to some problems with the measurement of the magnetic field some measurements to the influence of the Hall-Sensor were made. One with the sawtooth voltage and one without it. With these and the voltage and ampere used to create the magnetic field, the accuracy of measurement with the new Hall-Sensor can be determined more closely.