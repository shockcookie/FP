\section{Theory}
The contents of this chapter are, if not otherwise specified, derived from the guide to the experiment \cite{anleitung}
\subsection{Spin and nuclear spin}
The spin or intrinsic angular momentum of a elementary particle is an intrinsic property of particles from the family of the fermions. Members of this family, such as protons, neutrons and electrons all have a spin of $s=\frac{1}{2}$.
The spin can be explained semi classically, as rotation of the particle around its own 'center of gravity', with fixed frequency and variable axis of rotation. 
However, this illustration only makes sense in finite-size particles, of course. Just as with the angular momentum, not all three spin components can be defined at the same time, but only the amount and projection on a freely selectable 'quantization axis'.
