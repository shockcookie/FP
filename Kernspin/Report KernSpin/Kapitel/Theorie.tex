\section{Theory}
The contents of this chapter are, if not otherwise specified, derived from the guide to the experiment \cite{anleitung}
\subsection{Spin and nuclear spin}
The spin or intrinsic angular momentum of a elementary particle is an intrinsic property of particles from the family of the fermions. Members of this family, such as protons, neutrons and electrons all have a spin of $s=\frac{1}{2}$.
The spin can be explained semi classically, as rotation of the particle around its own 'centre of mass', with fixed frequency and variable axis of rotation. 
However, this illustration only makes sense in finite-size particles, of course. Just as with the angular momentum, not all three spin components can be defined at the same time, but only the amount and projection on a freely selectable 'quantization axis'.
The possible spin quantum numbers are $$ \left|\vec{S}\right| = \hbar \sqrt{S\left(S+1\right)}$$ with $S = 0, \frac{1}{2}, 1, ...$ and Planck's constant $\hbar$. 
Atomic nuclei are also assigned a spin, the nuclear spin, which is defined with the nuclear spin number $I$, analogue to the spin: 
$$\left|\vec{I}\right| = \hbar \sqrt{I\left(I+1\right)}$$
The nuclear spin number is also quantified in its direction. Analogue to the electron spin, the projection of the nuclear spin can also assume certain states as , e.g. with the z-axis as the quantization axis $I_z = m_I \hbar$ with $-I\le m_I \le + I$. In total there would be  $2I+1$ different states for $I_z$. Protons or the nucleus of $^{19}$F both have a nuclear spin number of $I=\frac{1}{2}$. So both have only two possible states: $m_I = \pm \frac{1}{2}$. They can only align parallel or antiparallel with the quantization axis in the experiment.

\subsection{Magnetic momentum}
The spin of a quantum mechanical particle is connected to a magnetic dipole momentum $\vec{\mu}$, the ratio of both is described as the gyromagnetic ratio $\gamma$.
$$\vec{\mu}=\gamma\vec{I}\qquad \textrm{with}\quad \gamma = \frac{g_I\mu_K}{\hbar}$$
The constant $g_I$ is the nuclear g factor, which is to be calculated during the exam. $g_I$ has no dimension and is unique for each nucleus. The second constant $\mu_K$ is the nuclear magneton, which is computed analogue to the Bohr magneton:
$$\mu_K = \frac{e\hbar}{2m_p}$$
The difference between those two is that for the Bohr magneton the elecron mass is used and for the nuclear magneton the proton mass. 
In the ground state of atomic nuclei, the nucleons are arranged according to the Pauli principle so that each orbital is occupied by two protons or neutrons of opposite spins. If now a eu-nucleus (with an even number of protons and an uneven number of neutrons) or if an ue-nucleus (where the even and uneven nucleons are reversed) is present, an unpaired nucleon remains. This leads to an half-digit total spin. For a uu-nucleus two unpaired nucleons remain resulting in an integer total spin. In a ee-nucleus all nucleons are paired, therefore the total spin is zero. Examples for ee-nuclei are $^{16}_{8}$O and $^{12}_6$C. Therefore it is possible to measure the spin of hydrogen $I=\frac{1}{2}$ utilizing glycol (C$_2$H$_6$O$_2$) and water (H$_2$O) samples. For the $^{9}_{19}$F nucleus with 9 protons and 10 neutrons the total spin is also $I=\frac{1}{2}$.
\subsection{Interaction with magnetic fields and radiation (nuclear magnetic resonance)}
